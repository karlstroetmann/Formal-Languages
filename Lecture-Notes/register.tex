\chapter{Register-Zuordnung}
Die \"Ubersetzung eines Integer-\texttt{C}-Programms in \textsl{Java}-Byte-Code ist deswegen einfach, weil
wir uns keine Gedanken dar\"uber machen m\"ussen, wie wir die Variablen auf Register verteilen, denn bei der
\textsc{Ijvm} gibt es keine f\"ur den Benutzer sichtbaren Register, so dass alle Variablen und auch die
Zwischenergebnisse auf dem Stack abgelegt werden m\"ussen.  In diesem Kapitel ist unser Ziel,
als Endergebnis ein Assembler-Programm f\"ur den \textsc{Srp}-Prozessor zu erzeugen.  Dazu m\"ussen wir
uns \"uberlegen, wie wir die einzelnen Variablen auf die verschiedenen Register verteilen.
Wir geben zun\"achst ein Verfahren an, mit dessen Hilfe wir die Anzahl der Register
bestimmen k\"onnen,  die f\"ur die Speicherung von Zwischenergebnissen, die bei der Auswertung
arithmetischer Ausdr\"ucke auftreten, ben\"otigt werden.
  Die Parameter und lokalen Variablen einer Funktion verteilen wir dann auf die
verbleibenden Register oder, falls nicht gen\"ugend Register vorhanden sind, allokieren wir
die lokalen Variablen auf dem Stack.

\section{Die Ershov-Zahlen}
Das Verfahren, mit dem Register f\"ur die Auswertung arithmetischer Ausdr\"ucke zugeordnet werden k\"onnen,
berechnet zun\"achst f\"ur einen gegebenen arithmetischen Ausdruck $e$ die Anzahl $\textsl{ershov}(e)$
der Register, die ben\"otigt werden, um den Ausdruck auszuwerten.  Es ist nach dem russischen
Informatiker A.~P.~Ershov benannt, der die Idee des  Verfahrens in \cite{ershov:58} publiziert hat.
Wir spezifizieren die Funktion
\\[0.2cm]
\hspace*{1.3cm}
$\textsl{ershov}: \mathtt{Expr} \rightarrow \mathbb{N}$,
\\[0.2cm]
die f\"ur einen gegebenen arithmetischen Ausdruck $e$ die Anzahl der zur Auswertung ben\"otigten
Register angibt, durch eine Fallunterscheidung nach dem Aufbau des Ausdrucks.
\begin{enumerate}
\item Um eine Konstante auszuwerten, ben\"otigen wir genau ein Register, in dem wir diese
      Konstante speichern k\"onnen:
      \\[0.2cm]
      \hspace*{1.3cm}
      $\textsl{isNumber}(c) \rightarrow \textsl{ershov}(c) = 1$.
\item Analog ben\"otigen wir zur Auswertung einer Variablen genau ein Register, in dem der Wert der
      Variablen abgelegt wird:
      \\[0.2cm]
      \hspace*{1.3cm}
      $\textsl{isVariable}(v) \rightarrow \textsl{ershov}(v) = 1$.
\item Zur Auswertung eines Ausdrucks der Form
      \\[0.2cm]
      \hspace*{1.3cm}
      $\textsl{op}\; e$,
      \\[0.2cm] 
      bei dem der un\"are Operator \textsl{op} auf den Ausdruck $e$ angewendet wird, ben\"otigen wir
      genau so viele Register, wie wir zur Auswertung des Ausdrucks $e$ ben\"otigen, denn wenn das
      Ergebnis der Auswertung von $e$ in einem Register $Rx$ liegt, so k\"onnen wir anschlie{\ss}end
      dieses Register \"uberschreiben um das Endergebnis der Auswertung abzuspeichern:
      \\[0.2cm]
      \hspace*{1.3cm}
      $\textsl{ershov}(\textsl{op}\; e) = \textsl{ershov}(e)$.
      \\[0.2cm]
      \textbf{Bemerkung}: Bei dem \textsc{Risc}-Prozessor, den wir in der Rechnertechnik-Vorlesung
      besprochen haben, hatten wir als einzigen un\"aren Operator die bitweise Negation, aber
      prinzipiell sind auch andere un\"are Operatoren denkbar.
\item Bei der Auswertung eines bin\"aren Ausdrucks der Form
      \\[0.2cm]
      \hspace*{1.3cm}
      $e_1 \;\textsl{op}\; e_2$
      \\[0.2cm]
      sind drei F\"alle zu unterscheiden.
      \begin{enumerate}
      \item $\textsl{ershov}(e_1) < \textsl{ershov}(e_2)$.

            In diesem Fall berechnen wir zun\"achst den Ausdruck $e_2$ und ben\"otigen dazu
            $\textsl{ershov}(e_2)$ Register.  Das letzte dieser Register, nennen wir es
            $R_y$,             enth\"alt dann das Ergebnis
            der Auswertung von $e_2$, alle anderen Register k\"onnen f\"ur die Berechnung von $e_1$
            verwendet werden.  Da $\textsl{ershov}(e_1) < \textsl{ershov}(e_2)$ ist, reichen diese
            Register auch aus und das Ergebnis von $e_1$ wird in einem dieser Register, sagen
            wir mal $R_x$, abgespeichert.  Anschlie{\ss}end f\"uhren wir dann noch den Befehl
            \\[0.2cm]
            \hspace*{1.3cm}
            \textsl{op} $R_y$, $R_x$, $R_y$ 
            \\[0.2cm]
            aus, bei dem wir das Ergebnis des Ausdrucks $e_1 \;\textsl{op}\; e_2$ berechnen und in
            dem Register $R_y$ abspeichern.  Wir sehen, dass wir in diesem Fall zur Berechnung
            von $e_1 \;\textsl{op}\; e_2$ nicht mehr Register ben\"otigen als zur Berechnung von $e_2$
            und halten die folgende Formel fest:
            \\[0.2cm]
            \hspace*{1.3cm}
            $\textsl{ershov}(e_1) < \textsl{ershov}(e_2) \rightarrow 
             \textsl{ershov}(e_1 \;\textsl{op}\; e_2) = \textsl{ershov}(e_2)$.
      \item $\textsl{ershov}(e_1) = \textsl{ershov}(e_2)$.

            In diesem Fall berechnen wir den Ausdruck $e_1$ und ben\"otigen daf\"ur
            $\textsl{ershov}(e_1)$ viele Register.  Bis auf das letzte dabei ben\"otigte Register, 
            nennen wir es $R_x$, in dem wir das Ergebnis dieser Rechnung speichern m\"ussen, k\"onnen
            wir alle diese Register f\"ur die Berechnung des Ausdrucks $e_2$ wiederverwenden.
            Folglich ben\"otigen wir f\"ur die Berechnung von $e_2$ also ein weiteres Register, das wir
            jetzt mit $R_y$ bezeichnen.  Das Endergebnis k\"onnen wir nun mit dem Befehl
            \\[0.2cm]
            \hspace*{1.3cm}
            \textsl{op} $R_y$, $R_x$, $R_y$ 
            \\[0.2cm]
            berechnen, so dass wir an dieser Stelle kein weiteres Register mehr ben\"otigen.
            Wir halten fest:
            \\[0.2cm]
            \hspace*{1.3cm}
            $\textsl{ershov}(e_1) = \textsl{ershov}(e_2) \rightarrow 
             \textsl{ershov}(e_1 \;\textsl{op}\; e_2) = \textsl{ershov}(e_1) + 1$.
      \item $\textsl{ershov}(e_1) > \textsl{ershov}(e_2)$.

            Dieser Fall ist offenbar analog zum ersten Fall und wir haben:
            \\[0.2cm]
            \hspace*{1.3cm}
            $\textsl{ershov}(e_1) > \textsl{ershov}(e_2) \rightarrow 
             \textsl{ershov}(e_1 \;\textsl{op}\; e_2) = \textsl{ershov}(e_1)$.
      \end{enumerate}
\item Die Auswertung eines Funktions-Aufrufs der Form
      \\[0.2cm]
      \hspace*{1.3cm}
      $f(e_1, \cdots, e_n)$
      \\[0.2cm]
      gelingt in Registern nur dann, wenn die einzelnen Ausdr\"ucke $e_1$, $\cdots$, $e_n$
      selber keine Funktions-Aufrufe mehr enthalten.  Wir werden daher unsere Grammatik so
      einschr\"anken, dass wir nur solche Funktions-Aufrufe betrachten, bei denen dies der
      Fall ist.  Des Weiteren werden wir Funktions-Aufrufe nur auf der rechten Seite einer
      Zuweisungen zulassen.  Dann k\"onnen wir jedes der Argumente von $f$ getrennt
      auswerten.  Da die Ergebnisse dieser Auswertungen ohnehin auf den Stack geschrieben
      werden, ist es nicht erforderlich, diese Ergebnisse in Registern zu speichern.
      Daher haben wir
      \\[0.2cm]
      \hspace*{1.3cm}
      $\textsl{ershov}\bigl(f(e_1, \cdots, e_n)\bigr) = 
       \textsl{max}\bigl( \textsl{ershov}(e_1), \cdots, \textsl{ershov}(e_n)\bigr)$.     
\end{enumerate}

\exercise
Geben Sie an, wie gro{\ss} ein Ausdruck $e$ mindestens sein muss, damit 
$\textsl{ershov}(e) = n$ gilt.



\section{Implementierung eines Compilers f\"ur den \textsc{SRP}}
Nach den Vor\"uberlegungen aus dem letzten Abschnitt entwickeln wir 
in diesem Abschnitt einen Compiler, der Code f\"ur den \textsc{Srp} erzeugt.
Wir gehen dabei wieder von der Sprache Integer-\texttt{C} aus, f\"ur die wir 
im letzten Kapitel einen Compiler in Byte-Code entwicklet haten.
  Diese Grammatik hatten wir in Abbildung \ref{fig:compiler.cup} gezeigt.

Die Idee bei der Entwicklung eines Compilers f\"ur den \textsc{Srp} ist, dass wir zun\"achst
berechnen, wieviele Register wir ben\"otigen, um die 
Zwischenergebnisse, die bei der Auswertung arithmetischer Ausdr\"ucke entstehen, zu
speichern.  Die restlichen Register k\"onnen dann f\"ur die lokalen Variablen genutzt werden.
Sollten wir mehr lokale Variablen als freie Register haben, so speichern wir die
\"uberz\"ahligen lokalen Variablen auf dem Stack.
Wir werden nur die Register \texttt{R1} bis \texttt{R29} benutzen, denn  wir gehen davon 
aus,  dass folgendes gilt:
\begin{enumerate}
\item Das Register \texttt{R0}  speichert die Konstante $0$.
\item Das Register \texttt{R30} speichert den Stack-Pointer.
\item Das Register \texttt{R31} ist f\"ur die Berechnung von Sprungadressen und
      Speicherstellen reserviert.  
\end{enumerate}
In der Rechnertechnik-Vorlesung hatten wir eine Unterscheidung in solche
Register, die beim Aufruf einer Funktion gesichert werden m\"ussen und solche Register, die
nicht gesichert werden, durchgef\"uhrt.  Diese Unterscheidung lassen wir fallen und
vereinbaren statt dessen, dass bei einem Funktionsaufruf alle Register, die von der
Funktion ver\"andert werden, gesichert werden m\"ussen.  
Das vereinfacht die Entwicklung des Compilers etwas.

\subsection{Diskussion der verschiedenen Klassen}
Die Struktur des \textsc{SRP}-Compilers ist \"ahnlich zu der Struktur des
\textsl{Java}-Byte-Code-Compilers:  Wir erzeugen zun\"achst einen abstrakten Syntax-Baum und
haben dann in allen Klassen eine Methode $\textsl{compile}()$, die Code erzeugt, der den jeweils
dargestellten Ausdruck \"ubersetzt.  Die Erzeugung des Codes ist allerdings deutlich
komplexer als bei dem Byte-Code-Compiler.


\begin{figure}[!ht]
\centering
\begin{Verbatim}[ frame         = lines, 
                  framesep      = 0.3cm, 
                  labelposition = bottomline,
                  numbers       = left,
                  numbersep     = -0.2cm,
                  xleftmargin   = 0.0cm,
                  xrightmargin  = 0.0cm,
                ]
    public class Function {
        private String            mName;
        private List<String>      mParameterList;
        private List<Declaration> mDeclarations;
        private List<Statement>   mBody;
        private Integer           mNumberUsedRegs;
        private Integer           mNumberVarsOnStack;
        private Map<String, Location> mSymbolTable;    
        public  static String sName;
    
        public Function(String            name, 
                        List<String>      parameterList, 
                        List<Declaration> declarations, 
                        List<Statement>   body) 
        {
            mName          = name;
            mParameterList = parameterList;
            mDeclarations  = declarations;
            mBody          = body;
        }
        public List<AssemblerCmd> compile() {
            initTable();
            List<AssemblerCmd> result = new LinkedList<AssemblerCmd>();
            result.add(new PROC(mName));
            result.addAll(saveRegisters());
            result.addAll(fillRegisters());
            for (Statement stmnt: mBody) {
                result.addAll(stmnt.compile(mSymbolTable));
            }
            result.add(new NEWLINE());
            return result;
        }
\end{Verbatim} 
\vspace*{-0.3cm} %$
\caption{Die Klasse \texttt{Function}, 1.~Teil.}
\label{fig:Function.java-SRP-1}
\end{figure}

\begin{figure}[!ht]
\centering
\begin{Verbatim}[ frame         = lines, 
                  framesep      = 0.3cm, 
                  firstnumber   = last,
                  labelposition = bottomline,
                  numbers       = left,
                  numbersep     = -0.2cm,
                  xleftmargin   = 0.0cm,
                  xrightmargin  = 0.0cm,
                ]
    private List<AssemblerCmd> saveRegisters() {
        List<AssemblerCmd> result = new LinkedList<AssemblerCmd>();
        for (int i = 1; i <= mNumberUsedRegs; ++i) {
            result.add(new STOREC(i, mNumberVarsOnStack + i));
        }
        int nextFree = mNumberVarsOnStack + mNumberUsedRegs + 1;
        mSymbolTable.put("$free@" + mName, 
                         new Memory(nextFree));
        mSymbolTable.put("$numberVarsOnStack@" + mName, 
                         new Memory(mNumberVarsOnStack));
        mSymbolTable.put("$numberUsedRegs@" + mName, 
                         new Memory(mNumberUsedRegs   ));
        return result;
    }
    private List<AssemblerCmd> fillRegisters() {
        List<AssemblerCmd> result = new LinkedList<AssemblerCmd>();
        for (int i = 0; i < mParameterList.size(); ++i) {
            String p = mParameterList.get(i);
            Location l = mSymbolTable.get(p);
            if (l instanceof Register) {
                Integer register = ((Register) l).getNumber();
                Integer offset   = mParameterList.size() - i;
                result.add(new LOADC(register, -offset));
            } else {
                return result;
            }
        }
        return result;
    }
\end{Verbatim}
\vspace*{-0.3cm} %\$
\caption{Die Klasse \texttt{Function}, 2.~Teil.}
\label{fig:Function.java-SRP-2}
\end{figure}


Wir beginnen unsere Diskussion des \textsc{Srp}-Compilers mit der Klasse
\texttt{Function}, die in den Abbildungen \ref{fig:Function.java-SRP-1},
\ref{fig:Function.java-SRP-2}, \ref{fig:Function.java-SRP-3} und
\ref{fig:Function.java-SRP-4} gezeigt ist.
Abbildung \ref{fig:Function.java-SRP-1} auf Seite \pageref{fig:Function.java-SRP-1} zeigt die
Member-Variablen der Klasse \texttt{Function}, den Konstruktor, sowie die Methode
$\textsl{compile}()$.  Die Klasse repr\"asentiert eine Funktions-Definition der Form
\\[0.2cm]
\hspace*{1.3cm}
\texttt{int $f$($p_1, \cdots, p_n$) \{ \textsl{decls}\ \textsl{body} \}}
\\[0.2cm]
Die Member-Variablen haben folgende Bedeutung:
\begin{enumerate}
\item \texttt{mName} entspricht dem Funktions-Namen $f$.
\item \texttt{mParameterList} entspricht den \"Ubergabe-Parametern $p_1, \cdots, p_n$.
\item \texttt{mDeclarations} ist die Liste der Deklarationen aller lokalen Variablen.
\item \texttt{mBody} ist die Liste der Befehle im Rumpf der Funktion.
\item \texttt{mNumberUsedRegs} ist die Anzahl der Register, die von dieser Funktion
      benutzt werden.
\item \texttt{mNumberVarsOnStack} ist die Anzahl der lokalen Variablen, die auf dem Stack
      gespeichert werden m\"ussen, weil in den Registern kein Platz mehr f\"ur diese Variablen ist.
\item \texttt{mSymbolTable} gibt f\"ur jede lokale Variable den Ort an, wo diese lokale
      Variable gespeichert ist.  Zur Spezifikation dieses Ortes haben wir die abstrakte
      Klasse \texttt{Location} eingef\"uhrt, die \"uber die Gleichung
      \\[0.2cm]
      \hspace*{1.3cm}
      \texttt{Location = Memory(Integer offset) + Register(Integer number);}
      \\[0.2cm]
      definiert ist.  Die Klasse \texttt{Location} unterscheidet also, ob die Variable in
      einem Register gespeichert ist, oder ob die Variable auf dem Stack liegt.
      Falls die Variable in einem Register gespeichert wird, dann gibt \texttt{mNumber}
      an, um welches Register es sich handelt.  Liegt die Variable auf dem Stack, so wird
      diese Variable an der Stelle
      \\[0.2cm]
      \hspace*{1.3cm}
      \texttt{SP + offset}
      \\[0.2cm]
      gespeichert.  Dabei gehen wir davon aus, dass der Stackpointer \texttt{SP} immer auf
      die R\"ucksprung-Adresse der Funktion zeigt, so dass wir die Adresse der lokalen
      Variablen tats\"achlich immer nach der obigen Formel berechnen k\"onnen.
\item Zus\"atzlich hat die Klasse eine statische Variable \texttt{sName}, die den Namen
      der Funktion enth\"alt, die gerade geparst wird.
\end{enumerate}
Der Konstruktor der Klasse \texttt{Function} speichert die Member-Variablen, die den
abstrakten Syntax-Baum repr\"asentieren und die bereits vom Parser erzeugt werden k\"onnen.

Interessanter ist die Methode $\textsl{compile}()$, die diese Funktion \"ubersetzt.
Im Gegensatz zu unserer Implementierung des \textsl{Java}-Byte-Code-Compilers behandeln
wir bei dieser Implementierung die Sonderf\"alle, die bei den Funktionen 
$\textsl{putchar()}$ und $\textsl{getchar}()$ auftreten, nicht sondern gehen davon aus,
dass diese Funktionen bereits in einer Bibliothek zur Verf\"ugung gestellt werden und dann
sp\"ater von einem Linker eingebunden werden.  Die Funktion $\textsl{compile}()$ arbeitet
dann wie folgt:
\begin{enumerate}
\item Zun\"achst berechnen wir f\"ur die Parameter und die lokalen Variablen der Funktion,
      wo diese abgespeichert werden.  Diese Berechnung wird von der Methode
      $\textsl{initTable}()$ durchgef\"uhrt.  Das Ergebnis dieser Rechnung wird
      in der Variablen \texttt{mSymbolTable} abgelegt.  Zus\"atzlich initialisiert diese
      Methode die Member-Variablen \texttt{mNumberUsedRegs} und \texttt{mNumberVarsOnStack}. 
\item Anschlie{\ss}end erzeugen wir eine Liste von Assembler-Befehlen, die mit der Assembler-Deklaration
      \\[0.2cm]
      \hspace*{1.3cm}
      \texttt{PROC} \textsl{mName}
      \\[0.2cm]
      beginnt.  Als erstes sichert die Prozedur dann alle die Register, die von der Prozedur
      eventuell ver\"andert werden.  Die daf\"ur notwendigen Assembler-Befehle werden von der
      Funktion $\textsl{saveRegisters}()$ erzeugt.
\item Danach werden die \"Ubergabe-Parameter, die wir in Registern speichern wollen,
      in die entsprechenden Register geladen.  Der daf\"ur notwendige Code wird von der
      Methode $\textsl{fillRegisters}()$ erzeugt.
\item Zum Schluss werden die einzelnen Befehle, die im Rumpf der Funktion stehen, \"ubersetzt.      
\end{enumerate}
Die Methode $\textsl{saveRegisters}()$ erzeugt Code, mit dem die Register, die in der
Methode benutzt werden, auf dem Stack gesichert werden.  Dazu benutzt Sie den
Pseudo-Assembler-Befehl
\\[0.2cm]
\hspace*{1.3cm}
\texttt{storec R$i$, SP + $o$},
\\[0.2cm]
mit dem das Register \texttt{R$i$} an die Adresse  $\texttt{SP} + o$ im Hauptspeicher
geschrieben wird.  Dieser Pseudo-Befehl wird in die folgenden Assembler-Befehle \"ubersetzt:

\noindent
\hspace*{1.3cm} \texttt{const R31, $o$} \\
\hspace*{1.3cm} \texttt{add \ \ R31, R30, R31} \\
\hspace*{1.3cm} \texttt{store R$i$, R31} 
\\[0.2cm]
Anschlie{\ss}end speichern wir die Adresse der ersten freien Stelle auf dem Stack in der
Symboltabelle unter dem Namen ``\symbol{36}free@'' + \textsl{name}, wobei \texttt{name}
f\"ur den Namen der Prozedur steht.
Genauso werden die Anzahl der Variablen, die wir auf dem Stack haben sowie die Anzahl der
benutzten Register in der Symboltabelle abgespeichert.  Diese Werte werden sp\"ater in der
Klasse \texttt{ReturnStmnt} ben\"otigt um die alten Werte der auf dem Stack gesicherten
Register wieder herzustellen.


Die Methode $\textsl{fillRegisters}()$ hat die Aufgabe, die Parameter, mit denen die
Funktion aufgerufen worden ist, vom Stack in die daf\"ur vorgesehenen Register zu laden.
Dazu benutzt diese Methode den Pseudo-Assembler-Befehl
\\[0.2cm]
\hspace*{1.3cm}
\texttt{loadc R$i$, SP + $o$},
\\[0.2cm]
mit dem das Register \texttt{R$i$} von der Adresse  $\texttt{SP} + o$ im Hauptspeicher
geladen wird.  Dieser Pseudo-Befehl wird in die folgenden Assembler-Befehle \"ubersetzt:
\\[0.2cm]
\noindent
\hspace*{1.3cm} \texttt{const R31, $o$} \\
\hspace*{1.3cm} \texttt{add \ \ R31, R30, R31} \\
\hspace*{1.3cm} \texttt{load \ R$i$, R31} 
\\[0.2cm]
\textbf{Bemerkung}: Reale Prozessoren haben in der Regel ein Kommando, das den obigen
Pseudobefehl direkt umsetzen kann.

\begin{figure}[!ht]
\centering
\begin{Verbatim}[ frame         = lines, 
                  framesep      = 0.3cm, 
                  firstnumber   = last,
                  labelposition = bottomline,
                  numbers       = left,
                  numbersep     = -0.2cm,
                  xleftmargin   = 0.0cm,
                  xrightmargin  = 0.0cm,
                ]
        private void initTable() {
            mSymbolTable = new TreeMap<String, Location>();
            int nextAvailable = maxErshov() + 1;
            for (int i = 0; i < mParameterList.size(); ++i) {
                String p = mParameterList.get(i);
                if (nextAvailable < 30) {
                    mSymbolTable.put(p, new Register(nextAvailable));
                    ++nextAvailable;
                } else {
                    int offset = - mParameterList.size() + i;
                    mSymbolTable.put(p, new Memory(offset));
                }
            }
            int offset = 0;    
            for (int i = 0; i < mDeclarations.size(); ++i) {
                String v = mDeclarations.get(i).getVar();
                if (nextAvailable < 30) {
                    mSymbolTable.put(v, new Register(nextAvailable));
                    ++nextAvailable;
                } else {
                    ++offset;
                    mSymbolTable.put(v, new Memory(offset));
                }
            }
            mNumberVarsOnStack = offset;
            mNumberUsedRegs    = nextAvailable - 1;
        }
\end{Verbatim}
\vspace*{-0.3cm}
\caption{Die Klasse \texttt{Function}, 3.~Teil.}
\label{fig:Function.java-SRP-3}
\end{figure}


Die in Abbildungen \ref{fig:Function.java-SRP-3} gezeigte Methode $\textsl{initTable}()$
berechnet die Symboltabelle, in der hinterlegt ist, wo die einzelnen Variablen
abgespeichert sind.  Dazu wird zun\"achst die Funktion $\textsl{maxErshov}()$ aufgerufen.
Diese Funktion berechnet, wieviele Register maximal f\"ur Zwischenergebnisse ben\"otigt
werden.  Das erste Register, \"uber das wir frei verf\"ugen k\"onnen, hat dann den Wert
\\[0.2cm]
\hspace*{1.3cm}
$\textsl{maxErshov}() + 1$.
\\[0.2cm]
Als n\"achstes ordnen wir in den Zeilen 65 bis 74 den \"Ubergabe-Parametern der Funktion
Register zu.  Falls wir f\"ur einen Parameter kein freies Register mehr finden k\"onnen,
ber\"ucksichtigen wir in Zeile 71, dass die Parameter einer Funktion ja schon auf dem Stack
unterhalb der R\"ucksprungadresse liegen und tragen die entsprechende Adresse in der
Symboltabelle ein.  Da auf Register wesentlich schneller zugegriffen werden kann als auf
den Stack, der ja im Hauptspeicher liegt, ist das
Ziel, m\"oglichst alle Parameter in Registern zur Verf\"ugung zu haben.  Wenn es aber nicht
gen\"ugend  Register gibt, die noch frei sind, dann m\"ussen wir uns eben damit begn\"ugen, die
\"uberz\"ahligen Parameter im Bedarfsfall vom Stack zu laden.


In den Zeilen 76 bis 85 suchen wir anschlie{\ss}end f\"ur die lokalen Variablen einen Platz.
Falls noch Register frei sind, ordnen wir den Variablen Register zu, sonst kommen die lokalen Variablen
auf den Stack.

\begin{figure}[!ht]
\centering
\begin{Verbatim}[ frame         = lines, 
                  framesep      = 0.3cm, 
                  firstnumber   = last,
                  labelposition = bottomline,
                  numbers       = left,
                  numbersep     = -0.2cm,
                  xleftmargin   = 0.0cm,
                  xrightmargin  = 0.0cm,
                ]
        private int maxErshov() {
            int result = 1;
            for (Statement stmnt: mBody) { 
                result = Math.max(result, stmnt.maxErshov());
            }
            return result;
        }
        public String getName() {
            return mName;
        }
        public List<String> getParameterList() {
            return mParameterList;
        }
        public List<Statement> getBody() {
            return mBody;
        }
        public String toString() {
            return "Function(" + mName + ", " + 
                   mParameterList + ", " + mBody + ")";
        }
    }
\end{Verbatim}
\vspace*{-0.3cm} %\$
\caption{Die Klasse \texttt{Function}, 4.~Teil.}
\label{fig:Function.java-SRP-4}
\end{figure}

Abbildung \ref{fig:Function.java-SRP-4} zeigt schlie{\ss}lich die Implementierung der Funktion
$\textsl{maxErshov}()$, bei der einfach das Maximum \"uber alle Befehle des Rumpfs der
Funktion gebildet wird.

\begin{figure}[!ht]
\centering
\begin{Verbatim}[ frame         = lines, 
                  framesep      = 0.3cm, 
                  labelposition = bottomline,
                  numbers       = left,
                  numbersep     = -0.2cm,
                  xleftmargin   = 0.0cm,
                  xrightmargin  = 0.0cm,
                ]
    import java.util.*;
    
    public class Sum extends Expr {
        private Expr mLhs;
        private Expr mRhs;
    
        public Sum(Expr lhs, Expr rhs) {
            mLhs = lhs;
            mRhs = rhs;
            if (mLhs.getErshov() < mRhs.getErshov()) {
                mErshov = mRhs.getErshov();
            } else if (mLhs.getErshov() > mRhs.getErshov()) {
                mErshov = mLhs.getErshov();
            } else {
                mErshov = mLhs.getErshov() + 1;
            }
        }
        public List<AssemblerCmd> compile(Integer base, 
                                          Map<String, Location> symbolTable) 
        {
            List<AssemblerCmd> result = new LinkedList<AssemblerCmd>();
            if (mLhs.getErshov() < mRhs.getErshov()) {
                List<AssemblerCmd> rhsCmds = mRhs.compile(base, symbolTable);
                List<AssemblerCmd> lhsCmds = mLhs.compile(base, symbolTable);
                result.addAll(rhsCmds);
                result.addAll(lhsCmds);
            } else if (mLhs.getErshov() > mRhs.getErshov()) {
                List<AssemblerCmd> lhsCmds = mLhs.compile(base, symbolTable);
                List<AssemblerCmd> rhsCmds = mRhs.compile(base, symbolTable);
                result.addAll(lhsCmds);
                result.addAll(rhsCmds);
            } else {
                List<AssemblerCmd> lhsCmds = mLhs.compile(base + 1, symbolTable);
                List<AssemblerCmd> rhsCmds = mRhs.compile(base    , symbolTable);
                result.addAll(lhsCmds);
                result.addAll(rhsCmds);
            }
            mRegister = base + mErshov;
            if (mRegister < 30) {
                result.add(new ADD(mRegister, mLhs.getReg(), mRhs.getReg()));
            } else {
                System.err.println("insufficient registers to compile:");
            }
            return result;
        }
        public Expr getLhs() { return mLhs; }
        public Expr getRhs() { return mRhs; }
    }
\end{Verbatim}
\vspace*{-0.3cm}
\caption{Die Klasse Sum.java}
\label{fig:Sum.java-SRP}
\end{figure}

Abbildung \ref{fig:Sum.java-SRP} zeigt exemplarisch, wie die Ershov-Zahlen f\"ur einzelne
Ausdr\"ucke berechnet werden und wie der Code f\"ur einen arithmetischen Ausdruck der Form
\\[0.2cm]
\hspace*{1.3cm}
$\textsl{lhs} + \textsl{rhs}$
\\[0.2cm]
produziert
wird.  Die Klasse \texttt{Sum} repr\"asentiert einen Ausdruck der Form
\\[0.2cm]
\hspace*{1.3cm}
$\textsl{mLhs} + \textsl{mRhs}$.
\\[0.2cm]
Von der abstrakten Klasse \textsl{Expr} erbt die Klasse \textsl{Sum} die beiden
Member-Variablen \texttt{mErshov} und \texttt{mRegister}.  Die Variable \texttt{mErshov}
gibt an, wieviele Register zur \"Ubersetzung dieses Ausdrucks ben\"otigt werden.  Die Variable
\texttt{mRegister} gibt das Register an, in dem der Wert dieses Ausdrucks abgespeichert wird.
Die Ershov-Zahl f\"ur diesen Ausdruck wird bereits im Konstruktor ermittelt, wobei die
Fallunterscheidung unmittelbar aus der im letzten Abschnitt gegebenen Diskussion folgt.

Die Methode $\textsl{compile}()$ bekommt als erstes Argument eine nat\"urliche Zahl
\textsl{base} und als zweites Argument die Symboltabelle.  Das Argument \textsl{base} gibt
dabei den Index des ersten Registers an, das f\"ur Zwischen-Ergebnisse genutzt werden kann.
\begin{enumerate}
\item Betrachten wir zun\"achst den Fall, dass die Zahl der Register, die f\"ur die Berechnung von
      \texttt{mLhs} ben\"otigt wird, kleiner ist, als die Zahl der f\"ur \texttt{mRhs} ben\"otigten
      Register.  In diesem Fall berechnen wir erst \texttt{mRhs}.  Anschlie{\ss}end berechnen
      wir \texttt{mLhs}.  Da wir daf\"ur weniger Register ben\"otigen als f\"ur die Berechnung
      von \texttt{mRhs} werden wir das Ergebnis, das wir bei der Berechnung von
      \texttt{mRhs} im letzten Register gesichert haben, nicht \"uberschreiben.

      Wir speichern f\"ur jeden Ausdruck in der Member-Variablen \texttt{mRegister}, die in
      der abstrakten Klasse \texttt{Expr} definiert ist und die \"uber die Funktion $\textsl{getReg}()$
      abgefragt werden kann, dasjenige Register, in dem das Ergebnis des Ausdrucks
      abgespeichert wird.  Wir setzen diese Variable in Zeile 38 und k\"onnen dann die Summe
      durch den Befehl
      \\[0.2cm]
      \hspace*{1.3cm}
      \texttt{add mRegister, mLhs.getReg(), mRhs.getReg()}
      \\[0.2cm]
      berechnen.  Dieser Fall wird in den Zeilen 22 -- 26 abgehandelt.
\item Der umgekehrte Fall, bei dem die Zahl der Register, die f\"ur die Berechnung von
      \texttt{mRhs} ben\"otigt wird, kleiner ist, als die Zahl der f\"ur \texttt{mLhs} ben\"otigten
      Register, kann analog behandelt werden und ist in den Zeilen
      27 -- 31 gezeigt.
\item Abschlie{\ss}end betrachten wir den Fall, in dem die Zahl der Register, die f\"ur die Berechnung von
      \texttt{mLhs} ben\"otigt wird, genauso gro{\ss} ist wie die Zahl der f\"ur \texttt{mRhs} ben\"otigten
      Register.  In diesem Fall benutzen wir zur Berechnung von \texttt{mLhs} die Register
      \\[0.2cm]
      \hspace*{1.3cm}
      $\textsl{base} + 1, \;\cdots,\; \textsl{base} + 1 + \textsl{mLhs}.\textsl{getErshov}()$.
      \\[0.2cm]
      Das Ergebnis dieser Rechnung ist dann in dem Register 
      $\textsl{base} + 1 + \textsl{mLhs}.\textsl{getErshov}()$ gespeichert.
      Wenn wir nun bei der Berechnung von \textsl{mRhs} bereits mit dem Register \textsl{base}
      starten, dann ist sicher gestellt, dass dabei das in 
      \\[0.2cm]
      \hspace*{1.3cm}
      $\textsl{base} + 1 + \textsl{mLhs}.\textsl{getErshov}()$ 
      \\[0.2cm]
      gespeicherte Ergebnis nicht \"uberschrieben werden kann.  
      Dieser Fall ist der Grund, warum wir bei der Methode $\textsl{compile}()$ das
      zus\"atzliche Argument \textsl{base} ben\"otigen.  Der Fall ist in den Zeilen 33 bis 36
      gezeigt.
\item Der Fall, dass die Anzahl der Register nicht ausreichend ist, um die
      Zwischen-Ergebnisse, die bei der Berechnung eines Ausdrucks auftreten k\"onnen, zu
      berechnen, wird in Zeile 42 behandelt.  Dieser Fall kann, wie in einer \"Ubungsaufgabe
      gezeigt wird, bei 29 frei verf\"ugbaren Registern de facto nicht auftreten.
\end{enumerate}

\begin{figure}[!ht]
\centering
\begin{Verbatim}[ frame         = lines, 
                  framesep      = 0.3cm, 
                  labelposition = bottomline,
                  numbers       = left,
                  numbersep     = -0.2cm,
                  xleftmargin   = 0.0cm,
                  xrightmargin  = 0.0cm,
                ]
    public class ReturnStmnt extends Statement {
        private Expr   mExpr;
    	private String mName;
    
        public ReturnStmnt(Expr expr, String name) {
            mExpr = expr;
            mName = name;
        }
        public List<AssemblerCmd> compile(Map<String, Location> symbolTable) {
            int numVarsOnStack = 
                ((Memory) symbolTable.get("$numberVarsOnStack@" + mName)).getOffset();
            int numUsedRegs = 
                ((Memory) symbolTable.get("$numberUsedRegs@" + mName)).getOffset();
            List<AssemblerCmd> result = mExpr.compile(0, symbolTable);
            result.add(new STOREC(mExpr.getReg(), 1));
            result.addAll(restoreRegisters(numVarsOnStack, numUsedRegs));
            result.add(new RETURN());
            return result;
        }
        private List<AssemblerCmd> restoreRegisters(int numVarsOnStack, int numUsedRegs) {
            List<AssemblerCmd> result = new LinkedList<AssemblerCmd>();
            for (int i = 1; i <= numUsedRegs; ++i) {
                result.add(new LOADC(i, numVarsOnStack + i));
            }
            return result;
        }
        protected int maxErshov() {
            return mExpr.getErshov();
        }
        public Expr getExpr() {
            return mExpr;
        }
        public String toString() {
            return "ReturnStmnt(" + mExpr.toString() + ")";
        }
    }
\end{Verbatim}
\vspace*{-0.3cm}
\caption{Die Klasse \texttt{ReturnStmnt}.}
\label{fig:ReturnStmnt.java-SRP}
\end{figure}

\noindent
Die  in Abbildung \ref{fig:ReturnStmnt.java-SRP} gezeigte Klasse \texttt{ReturnStmnt}
zeigt die \"Ubersetzung eines Befehls der Form 
\\[0.2cm]
\hspace*{1.3cm}
\texttt{return \textsl{expr};} 
\\[0.2cm]
In der Methode $\textsl{compile}()$ wird zun\"achst der Ausdruck \textsl{expr} berechnet und
das Ergebnis dieser Rechnung wird in Zeile 15 auf den Stack gelegt.
Anschlie{\ss}end werden mit dem von der Methode $\textsl{restoreRegisters}()$ erzeugten Code
die Register wieder auf die Werte zur\"uck gesetzt, die auf dem Stack gespeichert wurden.


\begin{figure}[!ht]
\centering
\begin{Verbatim}[ frame         = lines, 
                  framesep      = 0.3cm, 
                  labelposition = bottomline,
                  numbers       = left,
                  numbersep     = -0.2cm,
                  xleftmargin   = 0.0cm,
                  xrightmargin  = 0.0cm,
                ]
    public class FunctionCall extends Expr {
        private String     mName;
        private List<Expr> mArgs;
    
        public FunctionCall(String name, List<Expr> args) {
            mName = name;
            mArgs = args;
            mErshov = 1;
            for (Expr arg: mArgs) {
                mErshov = Math.max(arg.getErshov(), mErshov);
            }
        }
        public List<AssemblerCmd> 
            compile(Integer base, Map<String, Location> symbolTable) 
        {
            Location l = symbolTable.get("$free");
            Memory m = (Memory) l;
            Integer k = m.getOffset();
            List<AssemblerCmd> result = new LinkedList<AssemblerCmd>();
            for (Expr arg: mArgs) {
                result.addAll(arg.compile(base, symbolTable));
                result.add(new STOREC(arg.getReg(), k)); // store Rx, SP + k
                ++k;
            }
            mRegister = 1;
            result.add(new CONST(31, k + mArgs.size() - 1));
            result.add(new ADD(30, 30, 31));   // SP = SP + k + n - 1
            result.add(new CALL(mName));
            result.add(new LOADC(mRegister, 1));
            result.add(new CONST(31, - k - mArgs.size()));
            result.add(new ADD(30, 30, 31));   // SP = SP -(k + n)
            return result;
        }
        public String     getName() { return mName; }
        public List<Expr> getArgs() { return mArgs; }
    }
\end{Verbatim}
\vspace*{-0.3cm} %$
\caption{Die Klasse \texttt{FunctionCall}.}
\label{fig:FunctionCall.java-SRP}
\end{figure}

Zum Abschluss diskutieren wir die in Abbildung \ref{fig:FunctionCall.java-SRP} gezeigte 
 Klasse \texttt{FunctionCall}, die einen Funktions-Aufruf der Form
\\[0.2cm]
\hspace*{1.3cm}
$f(e_1, \cdots, e_n)$
\\[0.2cm]
repr\"asentiert.  Der Konstruktor initialisiert sowohl den Namen als auch die Argumente der
Funktion und berechnet anschlie{\ss}end die Ershov-Zahl als Maximum der Ershov-Zahlen der
Argumente.  Die Aufgabe der Methode \textsl{compile} ist es nun, Code zu erzeugen, der
diese Argumente berechnet und auf dem Stack ablegt.  In der Symboltabelle haben wir unter
der Pseudo-Variablen ``\symbol{36}free'' gespeichert, wo der erste freie Platz auf dem
Stack ist. Dort legen wir in der Schleife in Zeile 20 -- 23 die Argumente ab.
Anschlie{\ss}end erh\"ohen wir in Zeile 26 den Stack-Pointer, so dass er nun auf das letzte auf
dem Stack abgelegte Argument zeigt und rufen die Funktion mit einem \texttt{call}-Befehl auf.
In Zeile 29 laden wir das Ergebnis vom Stack und speichern es in dem daf\"ur vorgesehenen
Register.  Danach wird der Stack-Pointer wieder auf seinen urspr\"unglichen Wert zur\"uck gesetzt.

\section{Schlussbetrachtung}
Der im letzten Abschnitt dargestellte Compiler geht sehr sorglos mit Registern um.
Insbesondere das Sichern und Wiederherstellen der Register auf dem Stack, das von diesem
Compiler durchgef\"uhrt w\"urde, ist f\"ur die Praxis viel zu aufwendig.  Um einen
halbwegs brauchbaren Code zu erzeugen, sind wenigstens die folgenden Schritte
erforderlich:
\begin{enumerate}
\item Zun\"achst muss \"uberpr\"uft werden, welche Variablen an welcher Stelle des Programms
      \emph{lebendig} sind.  Dabei wird eine Variable an einer Stelle des Programms 
      dann als \emph{lebendig} bezeichnet, wenn der Wert, den diese Variable an der Stelle
      hat, sp\"ater noch f\"ur eine Rechnung ben\"otigt wird.

      Der Grund, warum eine Berechnung der lebendigen Variablen Sinn macht, ist der, dass
      ein Register, das von einer Variablen belegt wird, die an dieser Stelle des
      Programms nicht mehr lebendig ist, von einer anderen Variablen genutzt werden kann.
\item Anschlie{\ss}end geschieht nun die Register-Allokation, bei der nun auch verschiedenen
      Variablen dasselbe Register zugeordnet werden kann.   Wird zwei verschiedenen
      Variablen $x$ und $y$ dasselbe Register zugewiesen, so muss lediglich gew\"ahrleistet 
      werden, dass es keinen Punkt im Programm gibt, bei dem gleichzeitig sowohl $x$ als
      auch $y$ lebendig sind.
\item Bei dem Sichern und Wiederherstellen der Register auf dem Stack muss ausgehend von der Analyse
      der Lebendigkeit der Variablen garantiert werden, dass nur solche Register gesichert
      werden, deren Werte tats\"achlich sp\"ater noch ben\"otigt werden. 
\end{enumerate}
Wir werden uns im n\"achsten Kapitel n\"aher mit der Frage der Lebendigkeit von Variablen besch\"aftigen.


%%% Local Variables: 
%%% mode: latex
%%% TeX-master: "formale-sprachen"
%%% End: 
