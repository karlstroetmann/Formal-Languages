\chapter{Entwicklung eines einfachen Compilers} 
In diesem Kapitel konstruieren wir einen Compiler, der ein Fragment der Sprache \texttt{C}
in \textsl{Java}-Assembler �bersetzt.  
Das von dem Compiler �bersetzte Fragment der Sprache \texttt{C} bezeichnen wir als
\textsl{Integer}-\texttt{C}, denn es steht dort nur der Datentyp \texttt{int} zur Verf�gung.
Ein Compiler besteht prinzipiell aus den folgenden Komponenten:
\begin{enumerate}
\item Der Scanner liest die zu �bersetzende Datei ein und zerlegt diese in eine Folge von Token.
      
      Wir werden unseren Scanner mit Hilfe des Werkzeugs \textsl{JFlex} entwickeln.
\item Der Parser liest die Folge von Token und produziert als Ergebnis einen abstrakten Syntax-Baum.

      Wir werden den Parser mit Hilfe von \textsc{Cup} generieren.
\item Der Typ-Checker �berpr�ft den abstrakten Syntax-Baum auf Typ-Fehler.

      Da die von uns �bersetzte Sprache nur einen einzelnen Datentyp enth�lt, er�brigt
      sich diese Phase f�r den von uns entwickelten Compiler.
\item In realen Compilern erfolgt nun eine \emph{Optimierungsphase}, die wir aus Zeitgr�nden aber nicht betrachten.
\item Der Code-Generator �bersetzt schlie�lich den Parse-Baum in eine Folge von
      \textsc{Java}-Assembler-Befehlen.  
\end{enumerate}
Bei unseren Compiler sind wir an dieser Stelle schon fertig.  Bei Compilern, deren Zielcode ein
\textsc{Risc}-Assembler-Programm ist, wird normalerweise zun�chst auch ein Code erzeugt, der dem
\textsc{Jvm}-Code �hnelt.  Ein solcher Code wird als \emph{Zwischen-Code} bezeichnet.  Es bleibt
dann die Aufgabe eines sogenannten \emph{Backends}, daraus ein Assembler-Programm f�r einen gegebene
Prozessor-Architektur zur erzeugen.  Die schwierigste Aufgabe besteht hier darin, f�r die
verwendeten Variablen eine Register-Zuordnung zu finden, bei der m�glichst alle Variablen in
Registern vorgehalten werden k�nnen.  
Aus Zeitgr�nden k�nnen wir das Thema der Register-Zuordnung in dieser Vorlesung nicht behandeln.
% Dieses Thema ist allerdings so komplex, dass wir es in einem separaten Kapitel diskutieren.

\section{Die Programmiersprache \textsl{Integer}-\texttt{C}}

\begin{figure}[!ht]
  \begin{center}
  \begin{minipage}[t]{12.5cm}
  \begin{eqnarray*}    
\textsl{program}      & \rightarrow & \;\textsl{function}^* \\[0.2cm]
\textsl{function}     & \rightarrow & \quoted{int} \textsc{Id} \quoted{(} \textsl{paramList} \quoted{)} \quoted{\{} \textsl{decl}\,^*\; \textsl{stmnt}^* \;\squoted{\}} \\[0.2cm]
\textsl{paramList}    & \rightarrow & \;(\squoted{int}\, \textsc{Id}\; (\squoted{,}\, \quoted{int} \textsc{Id})^*)?  \\[0.2cm]
\textsl{decl}  & \rightarrow & \quoted{int} \textsc{Id} \quoted{;}  \\[0.2cm]
\textsl{stmnt}    & \rightarrow & \quoted{\{} \textsl{stmnt}^* \quoted{\}}  \\[0.0cm]
                      & \mid        & \;\textsc{Id} \quoted{=} \textsl{expr} \;\squoted{;} \\[0.0cm]
                      & \mid        &  \quoted{if} \quoted{(} \textsl{boolExpr} \quoted{)} \textsl{stmnt} \\[0.0cm]
                     & \mid         &  \quoted{if} \quoted{(} \textsl{boolExpr} \quoted{)} \textsl{stmnt} \quoted{else} \textsl{stmnt} \\[0.0cm]
                     & \mid        &  \quoted{while} \quoted{(} \textsl{boolExpr} \quoted{)} \textsl{stmnt} \\[0.0cm]
                     & \mid        &  \quoted{return} \textsl{expr}\; \squoted{;} \\[0.0cm]
                     & \mid        &  \;\textsl{expr} \;\squoted{;}               \\[0.2cm]
   \textsl{boolExpr} & \rightarrow & \textsl{expr} \;(\squoted{==} \mid \squoted{!=} \mid \squoted{<=} \mid \squoted{>=} \mid \squoted{<} \mid \squoted{>})\;  \textsl{expr} \\
         & \mid        &  \quoted{!} \textsl{boolExpr}             \\[0.0cm]
         & \mid        &  \textsl{boolExpr} \;(\squoted{\&\&} \mid \squoted{||})\; \textsl{boolExpr}   \\[0.2cm]
 \textsl{expr} & \rightarrow & \textsl{expr} \;(\squoted{+} \mid \squoted{-} \mid \squoted{*} \mid \squoted{/})\; \textsl{expr} \\[0.0cm]
     & \mid        &  \quoted{(} \textsl{expr} \quoted{)}                 \\[0.0cm]
     & \mid        &  \textsc{Number}                             \\[0.0cm]
     & \mid        &  \textsc{Id}\; (\squoted{(} (\textsl{expr}\; (\squoted{,}\; \textsl{expr})^*)?\;\squoted{)})?  
  \end{eqnarray*}
  \vspace*{-0.5cm}

  \end{minipage}
  \end{center}
  \caption{Eine \textsc{Ebnf}-Grammatik f�r \textsl{Integer}-\textsc{C}}
\label{fig:compiler.cup}
\end{figure}

\noindent
Wir stellen nun die Sprache \textsl{Integer}-\texttt{C} vor, die unser Compiler �bersetzen
soll.  In diesem Zusammenhang sprechen wir auch von der \emph{Quellsprache} unseres Compilers.
Abbildung \ref{fig:compiler.cup} zeigt die Grammatik der Quellsprache in 
erweiterter Backus-Naur-Form (\textsc{Ebnf}). 
Die Grammatik f�r \textsl{Integer}-\texttt{C} verwendet die folgenden beiden Terminale:
\begin{enumerate}
\item \textsc{Id} steht f�r eine Folge von Ziffern, Buchstaben und dem Unterstrich, die 
      mit einem Buchstaben beginnt.  Eine \textsc{Id} bezeichnet entweder eine Variable oder
      den Namen einer Funktion.
\item \textsc{Number} steht f�r eine Folge von Ziffern, die als Dezimalzahl interpretiert wird.
\end{enumerate}
Nach der oben angegebenen Grammatik ist ein Programm eine Liste von Funktionen.
Eine Funktion besteht aus der Deklaration der Signatur, worauf in geschweiften Klammern
eine Liste von Deklarationen (\textsl{decl}) und Befehlen (\textsl{stmt}) folgt.  Jeder Befehl wird
durch ein Semikolon abgeschlossen.  Der Aufbau der einzelnen Befehle ist dann
�hnlich wie bei der Sprache \textsc{Sl}, f�r die wir im Kapitel \ref{chapter:interpreter}
einen Interpreter entwickelt haben.
Die in Abbildung \ref{fig:compiler.cup} gezeigte Grammatik ist mehrdeutig:
\begin{enumerate}
\item Die Grammatik hat das \emph{Dangling-Else-Problem}.

      Da wir im Kapitel \ref{chapter:cup} bereits gesehen haben, wie dieses Problem
      professionell gel�st werden kann, nehme ich mir hier die Freiheit, \textsc{Cup} mit der Option
      \\[0.2cm]
      \hspace*{1.3cm}
      ``\texttt{-expect 1}
      \\[0.2cm]
      aufzurufen und dadurch die Fehlermeldung zu unterdr�cken, denn wir hatten ja bereits gesehen, das
      \textsc{Cup} per default den durch die Mehrdeutigkeit entstehenden Shift-Reduce-Konflikt in unserem
      Sinne aufl�st. 
\item F�r die bei arithmetischen und Boole'schen Ausdr�cken verwendeten Operatoren
      m�ssen Pr�zedenzen festgelegt werden.
\end{enumerate}

\begin{figure}[!ht]
\centering
\begin{Verbatim}[ frame         = lines, 
                  framesep      = 0.3cm, 
                  labelposition = bottomline,
                  numbers       = left,
                  numbersep     = -0.2cm,
                  xleftmargin   = 0.8cm,
                  xrightmargin  = 0.8cm,
                ]
    int sum(int n) {
        int s;
        s = 0;
        while (n != 0) {
            s = s + n;
            n = n - 1;    
        }
        return s;
    }  
    int main() {
        int n;
        n = 6 * 6;
        println(sum(n));
    }
\end{Verbatim}
\vspace*{-0.3cm}
\caption{Ein einfaches \textsc{Integer}-\texttt{C}-Programm.}
\label{fig:sum.c}
\end{figure}

\noindent
Abbildung \ref{fig:sum.c} zeigt ein einfaches \textsc{Integer}-\texttt{C}-Programm.  Die Funktion
$\textsl{sum}(n)$ berechnet die Summe
\\[0.2cm]
\hspace*{1.3cm}
 $\sum\limits_{i=1}^n i$
\\[0.2cm]
und die Funktion $\textsl{main}()$ ruft die Funktion \texttt{sum} mit dem Argument $6 \cdot 6$ auf.
Die in dem Programm verwendete Funktion \texttt{println} gibt ihr Argument gefolgt von einem
Zeilenumbruch aus.
\pagebreak

\section{Developing the  Scanner and the Parser}
We construct the scanner using  \textsl{JFlex} while we use \textsc{Cup} to implement the parser.
Figure \ref{fig:compiler.jflex} shows the scanner.  The scanner recognizes the operators, keywords,
identifiers, and numbers.  When we compare this scanner to the scanners we have discussed
previously, we realize that there is nothing here that we haven't seen before.  Therefore, we do not
discuss this scanner in more detail.


\begin{figure}[!ht]
\centering
\begin{Verbatim}[ frame         = lines, 
                  framesep      = 0.3cm, 
                  labelposition = bottomline,
                  numbers       = left,
                  numbersep     = -0.2cm,
                  xleftmargin   = 0.0cm,
                  xrightmargin  = 0.0cm,
                ]
    import java_cup.runtime.*;    
    %%
    %char
    %line
    %column
    %cupsym IntegerCParserSym
    %cup
    %unicode
    %{  private Symbol symbol(int type) {
            return new Symbol(type, yychar, yychar + yylength());
        }   
        private Symbol symbol(int type, Object value) {
            return new Symbol(type, yychar, yychar + yylength(), value);
        }
    %}
    %eofval{ return new Symbol(IntegerCParserSym.EOF);
    %eofval} 
    %%
    "+"                   { return symbol( IntegerCParserSym.PLUS      ); } 
    "-"                   { return symbol( IntegerCParserSym.MINUS     ); } 
    "*"                   { return symbol( IntegerCParserSym.TIMES     ); } 
    "/"                   { return symbol( IntegerCParserSym.SLASH     ); } 
    "("                   { return symbol( IntegerCParserSym.LPAREN    ); } 
    ")"                   { return symbol( IntegerCParserSym.RPAREN    ); }
    "{"                   { return symbol( IntegerCParserSym.LBRACE    ); }
    "}"                   { return symbol( IntegerCParserSym.RBRACE    ); }
    ","                   { return symbol( IntegerCParserSym.COMMA     ); }
    ";"                   { return symbol( IntegerCParserSym.SEMICOLON ); }
    "="                   { return symbol( IntegerCParserSym.ASSIGN    ); }
    "=="                  { return symbol( IntegerCParserSym.EQ        ); }
    "!="                  { return symbol( IntegerCParserSym.NE        ); }
    "<"                   { return symbol( IntegerCParserSym.LT        ); }
    ">"                   { return symbol( IntegerCParserSym.GT        ); }
    "<="                  { return symbol( IntegerCParserSym.LE        ); }
    ">="                  { return symbol( IntegerCParserSym.GE        ); }
    "&&"                  { return symbol( IntegerCParserSym.AND       ); }
    "||"                  { return symbol( IntegerCParserSym.OR        ); }
    "!"                   { return symbol( IntegerCParserSym.NOT       ); }
    "int"                 { return symbol( IntegerCParserSym.INT       ); }
    "return"              { return symbol( IntegerCParserSym.RETURN    ); }
    "if"                  { return symbol( IntegerCParserSym.IF        ); }
    "else"                { return symbol( IntegerCParserSym.ELSE      ); }
    "while"               { return symbol( IntegerCParserSym.WHILE     ); }   
    [a-zA-Z][a-zA-Z_0-9]* { return symbol(IntegerCParserSym.ID,     yytext());              }
    0|[1-9][0-9]*         { return symbol(IntegerCParserSym.NUMBER, new Integer(yytext())); }
    [ \t\v\n\r]           { /* skip white space          */ }   
    "//" [^\n]*           { /* skip single line comments */ }   
    "/*" ~"*/"            { /* skip multi  line comments */ }   
\end{Verbatim}
\vspace*{-0.3cm}
\caption{Der Scanner f�r \textsl{Integer}-\texttt{C}}
\label{fig:compiler.jflex}
\end{figure}


\begin{figure}[!ht]
\centering
\begin{Verbatim}[ frame         = lines, 
                  framesep      = 0.3cm, 
                  labelposition = bottomline,
                  numbers       = left,
                  numbersep     = -0.2cm,
                  xleftmargin   = 0.0cm,
                  xrightmargin  = 0.0cm,
                ]
    Program = Program(List<Function> functionList);
    
    Function = Function(String            name, 
                        List<String>      parameterList, 
                        List<Declaration> mDeclarations,
                        List<Statement>   body);
    
    Statement = Block(List<Statement> statementList)
              + Assign(String var, Expr expr)
              + IfThen(BoolExpr boolExpr, Statement statement)
              + IfThenElse(BoolExpr condition, Statement then, Statement else)
              + While(BoolExpr condition, Statement statement)
              + Return(Expr expr)
              + ExprStatement(Expr expr);
    
    Declaration = Declaration(String var);
    
    Expr = Sum(Expr lhs, Expr rhs)
         + Difference(Expr lhs, Expr rhs)
         + Product(Expr lhs, Expr rhs)
         + Quotient(Expr lhs, Expr rhs)
         + MyNumber(Integer number)
         + Variable(String name)
         + FunctionCall(String name, List<Expr> args);
    
    BoolExpr = Equation(Expr lhs, Expr rhs)
             + Inequation(Expr lhs, Expr rhs)
             + LessOrEqual(Expr lhs, Expr rhs)
             + GreaterOrEqual(Expr lhs, Expr rhs)
             + LessThan(Expr lhs, Expr rhs)
             + GreaterThan(Expr lhs, Expr rhs)
             + Negation(BoolExpr expr)
             + Conjunction(BoolExpr lhs, BoolExpr rhs)
             + Disjunction(BoolExpr lhs, BoolExpr rhs);
\end{Verbatim}
\vspace*{-0.3cm}
\caption{Spezifikation der Klassen zur Darstellung des Syntax-Baums.}
\label{fig:program.ep}
\end{figure}
Before we can discuss the parser, we have to specify the classes that are used to represent the
abstract syntax tree that is generated by the parser.  I have chosen to describe these classes using
a formal notation.  The specification is shown in Figure \ref{fig:program.ep} on page \pageref{fig:program.ep}.
We discuss this specification now line by line.
\begin{enumerate}
\item In line 1, the equation
      \\[0.2cm]
      \hspace*{1.3cm}
      \texttt{Program = Program(List<Function> functionList)}
      \\[0.2cm]
      specifies that the class \texttt{Program} has one member variable called
      \texttt{mFunctionList}.  (Note that it is called ``\texttt{mFunctionList}'' rather than
      ``\texttt{functionList}''.  The reason is that in order to distinguish the member variables of
      a class from the local variables of a method I have decided to start all member variables with
      the letter ``\texttt{m}''.)
      The variable \texttt{mFunctionList} has the type \texttt{List<Function>}.
      Therefore, an object of class \texttt{Program} is essentially a list of objects of
      class \texttt{Function}.
\item Similarly, in line 3 to 6 the equation
      \begin{verbatim}
    Function = Function(String            name, 
                        List<String>      parameterList, 
                        List<Declaration> mDeclarations,
                        List<Statement>   body);
      \end{verbatim}
      \vspace*{-0.5cm}

      specifies that the class \texttt{Function} has four attributes:
      \begin{enumerate}
      \item \texttt{mName} is a \texttt{String} storing the name of the function,
      \item \texttt{mParameterList} is the list of formal parameters of the function,
      \item \texttt{mDeclarations} is the list of local variable declarations occurring in
            the body of the function, while
      \item \texttt{mBody} is the list of statements that make up the definition of the function.
      \end{enumerate}
      \pagebreak

\item Next, the equation
      \begin{verbatim}
    Statement = Block(List<Statement> statementList)
              + Assign(String var, Expr expr)
              + IfThen(BoolExpr boolExpr, Statement statement)
              + IfThenElse(BoolExpr condition, Statement then, Statement else)
              + While(BoolExpr condition, Statement statement)
              + Return(Expr expr)
              + ExprStatement(Expr expr);
      \end{verbatim}
      tells us that there is an abstract class \texttt{Statement} and that the classes
      \texttt{Block}, \texttt{Assign}, $\cdots$, \texttt{ExprStatement} are derived from
      this class.
      \begin{enumerate}
      \item \texttt{Block} is a class representing a list of statements enclosed in
            the curly braces \squoted{\{} and \squoted{\}}.  This list of statemments is
            stored in the member variable \texttt{mStatementList}.
      \item \texttt{Assign} is a class representing an assignment statement.  Therefore,
            this class has two attributes:
            \begin{itemize}
            \item \texttt{mVar} is the name of the variable on the left hand side of the
                  assignment and
            \item \texttt{mExpr} is the expression on the right hand side of the  assignment.
            \end{itemize}
      \item \texttt{IfThen} is a class representing an \texttt{if-then} statement without
            a trailing \texttt{else} clause.  This class has two member variables:
            \begin{itemize}
            \item \texttt{mBoolExpr} is the Boolean expression controlling whether the
            \item \texttt{mStatement} is to be executed.
            \end{itemize}
      \item \texttt{IfThenElse} is a class representing an \texttt{if-then-else}
            statement. This class has three member variables:
            \begin{itemize}
            \item \texttt{mBoolExpr} is the Boolean expression controlling the execution.
            \item \texttt{mThen} is the statement that is executed if
                  \texttt{mBoolExpr} evaluates as \texttt{true}.
            \item \texttt{mElse} is the statement that is executed if
                  \texttt{mBoolExpr} evaluates as \texttt{false}.
            \end{itemize}
      \item \texttt{While} is a class representing a \texttt{while} loop. 
            This class has two member variables:
            \begin{itemize}
            \item \texttt{mBoolExpr} is the Boolean expression controlling the loop.
            \item \texttt{mStatement} is a statement that is executed as long as
                  \texttt{mBoolExpr} evaluates to \texttt{true}.
            \end{itemize}
      \item \texttt{Return} is a class representing a \texttt{return} statement. 
            This class has the member variable \texttt{mExpr}.  Evaluation of this expression
            yields the value to be returned.  Note that in the grammar given in Figure
            \ref{fig:compiler.cup} on page \pageref{fig:compiler.cup} the expression following the
            return statement is not optional.
      \item \texttt{ExprStatement} is a class representing an expression that is to be
             evaluated as a statement.        
      \end{enumerate}
\item The equation in line 16 specifies that the class \texttt{Declaration} has one member
      variable with name \texttt{mVar}.  This variable stores the name of the variable
      that is declared in the variable declaration associated with the corresponding
      object of class \texttt{Declaration}.
\item Similarly, the equations defining \texttt{Expr} and \texttt{BoolExpr} specify the
      representation of arithmetical and Boolean expressions.  We refrain from a detailed discussion
      here, since by now you should have gotten the idea.
\end{enumerate}
Now we are ready to present the \textsc{Cup} specification of our parser.  For reasons of space,
we have split the \textsc{Cup} grammar into three parts. 
\begin{enumerate}
\item Figure \ref{fig:compiler.cup-1} shows the specification of the syntactical variables,
      terminals, and the specification of the precedences of the terminals.
      Note that the arithmetical operators have a higher precedence than the logical operators.
\item Figure \ref{fig:compiler.cup-2} and \ref{fig:compiler.cup-3} show the grammar including the
      actions needed to construct the syntax tree.  Note that the grammar allows the list of
      functions to be empty.  Of course, an empty program wouldn't be of any use, but this approach
      simplifies the construction of the associated lists.

      Observe that the actions given in the grammar only construct the abstract syntax tree.
      Everything else, in particular the generation of code, is then delegated to appropriate
      methods in the classes representing the syntax tree.   We will discuss the code generation
      later in a separate section. 
\end{enumerate}



\begin{figure}[!ht]
\centering
\begin{Verbatim}[ frame         = lines, 
                  framesep      = 0.3cm, 
                  labelposition = bottomline,
                  numbers       = left,
                  numbersep     = -0.2cm,
                  xleftmargin   = 0.0cm,
                  xrightmargin  = 0.0cm,
                ]
    import java_cup.runtime.*;
    import java.util.*;
    
    class  IntegerCParser;
    
    terminal           COMMA, PLUS, MINUS, TIMES, SLASH, LPAREN, RPAREN, LBRACE, RBRACE;
    terminal           ASSIGN, EQ, LT, GT, LE, GE, NE, AND, OR, NOT;
    terminal           IF, ELSE, WHILE, RETURN, SEMICOLON; 
    terminal           INT;
    terminal String    ID;
    terminal Integer   NUMBER;
    
    nonterminal Program           program;
    nonterminal List<Function>    functionList;
    nonterminal Function          function;
    nonterminal List<String>      paramList, neParamList;
    nonterminal Declaration       declaration;
    nonterminal List<Declaration> declarations;
    nonterminal Statement         stmnt;
    nonterminal List<Statement>   stmntList;
    nonterminal Expr              expr;
    nonterminal List<Expr>        exprList, neExprList;
    nonterminal BoolExpr          boolExpr;
    
    precedence left     OR;
    precedence left     AND;
    precedence right    NOT;
    precedence left     PLUS, MINUS;
    precedence left     TIMES, SLASH;
\end{Verbatim}
\vspace*{-0.3cm}
\caption{Declaration of terminals, syntactical variables, and operator precedences.}
\label{fig:compiler.cup-1}
\end{figure}

\begin{figure}[!ht]
\centering
\begin{Verbatim}[ frame         = lines, 
                  framesep      = 0.3cm, 
                  firstnumber   = last,
                  labelposition = bottomline,
                  numbers       = left,
                  numbersep     = -0.2cm,
                  xleftmargin   = 0.0cm,
                  xrightmargin  = 0.0cm,
                ]
    start with program;
    
    program ::= functionList:l {: RESULT = new Program(l); :} ;
    
    functionList ::= /* epsilon */ {: RESULT = new LinkedList<Function>(); :}
                  |  functionList:l function:f {: l.add(f); RESULT = l; :}
                  ;
    
    function ::= INT ID:f LPAREN paramList:p RPAREN 
                 LBRACE declarations:d stmntList:l RBRACE
                 {: RESULT = new Function(f, p, d, l); :}
              ;
    
    paramList ::= /* epsilon */ {: RESULT = new LinkedList<String>(); :}
               |  neParamList:l {: RESULT = l;                        :}
               ;
    
    neParamList ::= INT ID:v {: RESULT = new LinkedList<String>(); RESULT.add(v); :}
                 |  neParamList:l COMMA INT ID:v {: RESULT = l; RESULT.add(v);    :}
                 ;           
    
    declaration ::= INT ID:v SEMICOLON {: RESULT = new Declaration(v); :} ;
    
    declarations ::= /* epsilon */ {: RESULT = new LinkedList<Declaration>(); :}
                  |  declarations:l declaration:d {: RESULT = l; RESULT.add(d); :}
                  ;
    
    stmnt ::= LBRACE stmntList:l RBRACE           {: RESULT = new Block(l);     :}
           |  ID:v ASSIGN expr:e SEMICOLON        {: RESULT = new Assign(v, e); :}
           |  IF LPAREN boolExpr:b RPAREN stmnt:s {: RESULT = new IfThen(b, s); :}        
           |  IF LPAREN boolExpr:b RPAREN stmnt:t ELSE stmnt:e
                  {: RESULT = new IfThenElse(b, t, e); :}
           |  WHILE LPAREN boolExpr:b RPAREN stmnt:s
                  {: RESULT = new While(b, s); :}
           |  RETURN expr:e SEMICOLON   {: RESULT = new Return(e);        :}
           |  expr:e SEMICOLON          {: RESULT = new ExprStatement(e); :}
           ;
    
    stmntList ::= /* epsilon */       {: RESULT = new LinkedList<Statement>(); :}
               |  stmntList:l stmnt:s {: l.add(s); RESULT = l;                 :}
               ;
\end{Verbatim}
\vspace*{-0.3cm}
\caption{The first part of the \textsc{Cup} grammar.}
\label{fig:compiler.cup-2}
\end{figure}


\begin{figure}[!ht]
\centering
\begin{Verbatim}[ frame         = lines, 
                  framesep      = 0.3cm, 
                  firstnumber   = last,
                  labelposition = bottomline,
                  numbers       = left,
                  numbersep     = -0.2cm,
                  xleftmargin   = 0.0cm,
                  xrightmargin  = 0.0cm,
                ]
    boolExpr ::= expr:l EQ expr:r          {: RESULT = new Equation(      l, r); :}
              |  expr:l NE expr:r          {: RESULT = new Inequation(    l, r); :}
              |  expr:l LE expr:r          {: RESULT = new LessOrEqual(   l, r); :}
              |  expr:l GE expr:r          {: RESULT = new GreaterOrEqual(l, r); :}
              |  expr:l LT expr:r          {: RESULT = new LessThan(      l, r); :}
              |  expr:l GT expr:r          {: RESULT = new GreaterThan(   l, r); :}
              |  NOT boolExpr:e            {: RESULT = new Negation(      e   ); :}
              |  boolExpr:l AND boolExpr:r {: RESULT = new Conjunction(   l, r); :}
              |  boolExpr:l OR  boolExpr:r {: RESULT = new Disjunction(   l, r); :}
              ;
    
    expr ::= expr:l PLUS  expr:r            {: RESULT = new Sum(       l, r);   :}
          |  expr:l MINUS expr:r            {: RESULT = new Difference(l, r);   :}
          |  expr:l TIMES expr:r            {: RESULT = new Product(   l, r);   :}
          |  expr:l SLASH expr:r            {: RESULT = new Quotient(  l, r);   :}
          |  LPAREN expr:e RPAREN           {: RESULT = e;                      :}
          |  NUMBER:n                       {: RESULT = new MyNumber(n);        :}
          |  ID:v                           {: RESULT = new Variable(v);        :}
          |  ID:n LPAREN exprList:l RPAREN  {: RESULT = new FunctionCall(n, l); :}
          ;
    
    exprList ::= /* epsilon */ {: RESULT = new LinkedList<Expr>(); :}
              |  neExprList:l  {: RESULT = l;                      :}
              ;
    
    neExprList ::= expr:e {: RESULT = new LinkedList<Expr>(); RESULT.add(e); :}                  
                |  neExprList:l COMMA expr:e {: RESULT = l; RESULT.add(e);   :}
                ;
\end{Verbatim}
\vspace*{-0.3cm}
\caption{The last part of the \textsc{Cup} grammar.}
\label{fig:compiler.cup-3}
\end{figure}



%%% Local Variables: 
%%% mode: latex
%%% TeX-master: "formal-languages"
%%% End: 
