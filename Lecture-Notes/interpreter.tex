\chapter{Interpreter \label{chapter:interpreter}}


\begin{figure}[!ht]
\centering
\begin{Verbatim}[ frame         = lines, 
                  framesep      = 0.3cm, 
                  firstnumber   = 1,
                  labelposition = bottomline,
                  numbers       = left,
                  numbersep     = -0.2cm,
                  xleftmargin   = 0.8cm,
                  xrightmargin  = 0.8cm,
                ]
    grammar Pure;
    
    program  : statement+
             ;  
    statement: VAR ':=' expr ';'
             | VAR ':=' 'read' '(' ')' ';'
             | 'print' '(' expr ')' ';'
             | 'if'    '(' boolExpr ')' '{' statement* '}'
             | 'while' '(' boolExpr ')' '{' statement* '}'
             ;
    boolExpr : expr '==' expr
             | expr '<'  expr
             ;
    expr     : expr '+' product
             | expr '-' product
             | product
             ;
    product  : product '*' factor
             | product '/' factor
             | factor
             ;
    factor   : '(' expr ')'
             | VAR
             | NUMBER
             ;
    VAR      : [a-zA-Z][a-zA-Z_0-9]*;
    NUMBER   : '0'|[1-9][0-9]*;
    MULTI_COMMENT : '/*' .*? '*/' -> skip;
    LINE_COMMENT  : '//' ~('\n')* -> skip;
    WS            : [ \t\v\n\r]   -> skip; 
\end{Verbatim}
\vspace*{-0.3cm}
\caption{\textsc{Antlr}-Grammatik f\"ur eine einfache Programmier-Sprache.}
\label{fig:Pure.g4}
\end{figure}

In diesem Kapitels erstellen wir mit Hilfe des Parser-Generators \textsc{Antlr} einen
Interpreter f\"ur eine einfache Programmiersprache.  Erfreulicherweise akzeptiert
\textsc{Antlr} seit der Version \texttt{4.0} auch Grammatiken, die einfache Links-Rekursion
enthalten.  Auch eine Links-Faktorisierung der Grammatik ist nicht mehr notwendig.
Abbildung \ref{fig:Pure.g4} zeigt die \textsc{Antlr}-Grammatik der Programmier-Sprache, f\"ur die wir
in diesem Abschnitt einen Interpreter entwickeln.  Die Befehle dieser Sprache sind Zuweisungen, Print-Befehle,
\texttt{if}-Abfragen, sowie \texttt{while}-Schleifen.  Abbildung \ref{fig:sum.sl} zeigt
ein Beispiel-Programm, das dieser Grammatik entspricht.  Dieses Programm liest zun\"achst eine Zahl
ein, die in der Variablen \texttt{n} gespeichert wird.  Anschlie{\ss}end wird die Summe
\\[0.2cm]
\hspace*{1.3cm}
$\sum\limits_{i=1}^{n^2} i$
\\[0.2cm]
in der Variablen \texttt{s} akkumuliert und am Ende des Programms ausgegeben.


\begin{figure}[!ht]
\centering
\begin{Verbatim}[ frame         = lines, 
                  framesep      = 0.3cm, 
                  labelposition = bottomline,
                  numbers       = left,
                  numbersep     = -0.2cm,
                  xleftmargin   = 0.8cm,
                  xrightmargin  = 0.8cm,
                ]
    n := read();
    s := 0;
    i := 0;
    while (i < n * n) {
        i := i + 1;
        s := s + i;
    }
    print(s);
\end{Verbatim}
\vspace*{-0.3cm}
\caption{Ein Programm zur Berechnung der Summe $\sum\limits_{i=0}^{n^2} i$.}
\label{fig:sum.sl}
\end{figure}



Um einen Interpreter f\"ur diese Sprache entwickeln zu k\"onnen, ben\"otigen wir
zun\"achst Klassen, mit denen wir die einzelnen Befehle darstellen k\"onnen.
Wir beginnen mit der abstrakten Klasse \texttt{Statement}.  Diese Klasse ist in Abbildung
\ref{fig:Statement.java} gezeigt und dient dazu,
Anweisungen unserer Programmier-Sprache darzustellen.
Wir werden von der Klasse \texttt{Statement} sp\"ater die Klassen
\texttt{Assignment}, \texttt{Print}, \texttt{IfThen} und \texttt{While} ableiten.
Die Klasse \texttt{Statement} ist selber abstrakt und enth\"alt 
im Wesentlichen die Deklaration der abstrakten Methode $\texttt{execute}()$.  Diese Methode
k\"onnen wir sp\"ater benutzen, um einen Befehle ausf\"uhren.
Zus\"atzlich speichern wir hier das Flag \texttt{isInteractive} als statische Variable.  Mit
diesem Flag steuern wir, ob der Interpreter interaktiv in einer Kommandozeile betrieben
wird, oder ob im Batch-Modus eine Datei abgearbeitet werden soll.  Au{\ss}erdem haben wir in der
Klasse \texttt{Statement} noch die statische Methode $\texttt{prompt}()$.  Diese wird nur
dann benutzt, wenn der Interpreter interaktiv von der Kommandozeile aus betrieben wird.
In diesem Fall gibt die Methode den folgenden Prompt aus:
\\[0.2cm]
\hspace*{1.3cm}
\texttt{SL> }
\\[0.2cm]
Dieser Prompt signalisiert dem Benutzer, dass der Interpreter auf eine Eingabe wartet.

\begin{figure}[!ht]
\centering
\begin{Verbatim}[ frame         = lines, 
                  framesep      = 0.3cm, 
                  labelposition = bottomline,
                  numbers       = left,
                  numbersep     = -0.2cm,
                  xleftmargin   = 0.8cm,
                  xrightmargin  = 0.8cm,
                ]
    public abstract class Statement {
        static boolean isInteractive = false;
        
        public abstract void execute();
    
        static void prompt() {
            if (isInteractive) {
                System.out.print("SL> ");
                System.out.flush();
            }
        }
    }
\end{Verbatim}
\vspace*{-0.3cm}
\caption{Die abstrakte Klasse \texttt{Statement}}
\label{fig:Statement.java}
\end{figure}

\begin{figure}[!ht]
\centering
\begin{Verbatim}[ frame         = lines, 
                  framesep      = 0.3cm, 
                  labelposition = bottomline,
                  numbers       = left,
                  numbersep     = -0.2cm,
                  xleftmargin   = 0.8cm,
                  xrightmargin  = 0.8cm,
                ]
    public class Assignment extends Statement {
        Variable mLhs;
        Expr     mRhs;
        
        public Assignment(Variable lhs, Expr rhs) {
            mLhs = lhs;
            mRhs = rhs;
        }
        public void execute() {
            Expr.sValueTable.put(mLhs.mName, mRhs.eval());
        }
    }
\end{Verbatim}
\vspace*{-0.3cm}
\caption{Die Klasse \texttt{Assignment}.}
\label{fig:Assignment.java}
\end{figure}


\vspace*{\fill} \pagebreak

Abbildung \ref{fig:Assignment.java} zeigt die Implementierung der Klasse \texttt{Assignment}.  Da
diese Klasse eine Zuweisung der Form
\\[0.2cm]
\hspace*{1.3cm} $\textsl{var} \;\mathtt{:=}\; \textsl{expr}$
\\[0.2cm]
darstellt, bei der einer Variablen \textsl{var} der Wert eines arithmetischen Ausdrucks
\textsl{expr} zugewiesen wird, hat diese Klasse zwei Member-Variablen um die Variable und den
Ausdruck zu speichern.  
\begin{enumerate}
\item Die erste Member-Variable ist \texttt{mLhs}.  Diese
      Member-Variable entspricht der Variablen auf der linken Seite des Zuweisungs-Operators
      ``\texttt{:=}''.
\item Die zweite Member-Variable ist \texttt{mRhs}.  Hier wird der arithmetische Ausdruck,
      der auf der rechten Seite des Zuweisungs-Operators steht, kodiert.  Diese
      Member-Variable hat den Typ \texttt{Expr}.  Hierbei handelt es sich um eine
      abstrakte Klasse zur Darstellung arithmetischer Ausdr\"ucke, von der wir sp\"ater
      konkrete Klassen ableiten.  Diese Klasse besitzt eine abstrakte Methode
      $\texttt{eval}()$, mit der ein arithmetischer Ausdruck ausgewertet werden kann.
\end{enumerate}
In der Klasse \texttt{Assignment} wertet die Methode $\texttt{execute}()$ den Ausdruck,
der in der Variablen \texttt{mRhs} gespeichert wird, mit Hilfe  der f\"ur Objekte der Klasse
\texttt{Expr} zur Verf\"ugung stehenden 
Methode $\texttt{eval}()$ aus und speichert den erhaltenen Wert in der Hashtabelle
\texttt{sValueTable} unter dem Namen der Variablen ab. 
Es handelt sich bei dieser Tabelle um eine sogenannte \emph{Symboltabelle}, in der die
Werte der einzelnen Variablen abgelegt werden.
Die Tabelle ist als statische Variable in der Klasse \texttt{Expr} definiert.
Diese Klasse wird in Abbildung \ref{fig:Expr.java} auf Seite \pageref{fig:Expr.java} gezeigt.


\begin{figure}[!ht]
\centering
\begin{Verbatim}[ frame         = lines, 
                  framesep      = 0.3cm, 
                  firstnumber   = 1,
                  labelposition = bottomline,
                  numbers       = left,
                  numbersep     = -0.2cm,
                  xleftmargin   = 0.8cm,
                  xrightmargin  = 0.8cm,
                ]
    public class Read extends Statement {
        Variable mLhs;
        
        public Read(Variable lhs) {
            mLhs = lhs;
        }
        public void execute() {
            System.out.print("> "); // write prompt
            System.out.flush();
            Scanner scanner = new Scanner(System.in);
            Double  value   = scanner.nextDouble();
            Expr.sValueTable.put(mLhs.mName, value);
        }    
    }
\end{Verbatim}
\vspace*{-0.3cm}
\caption{Die Klasse \texttt{Read}.}
\label{fig:Read.java}
\end{figure}

Abbildung \ref{fig:Read.java} zeigt die Implementierung der Klasse \texttt{Read}.  Diese
 Klasse stellt  eine Zuweisung der Form
\\[0.2cm]
\hspace*{1.3cm} \texttt{\textsl{var} = read();}
\\[0.2cm]
dar.  Daher besitzt diese Klasse eine Member-Variable \texttt{mVar}, in welcher das Ergebnis
der Lese-Operation abgespeichert wird.  
Die Methode $\texttt{execute}()$ gibt zun\"achst den String ``\texttt{> }'' als
Eingabe-Aufforderung aus.  Anschlie{\ss}end wird ein Objekt der in \textsl{Java}
vordefinierten Klasse \texttt{Scanner} erzeugt.  Diese Klasse stellt die Methode
$\texttt{nextDouble}()$ zur Verf\"ugung, mit deren Hilfe eine Flie{\ss}komma-Zahl eingelesen
werden kann.  Diese wird dann in der Symboltabelle unter dem Namen der Variablen,
der auf der linken Seite der Zuweisung steht, abgespeichert.



\begin{figure}[!ht]
\centering
\begin{Verbatim}[ frame         = lines, 
                  framesep      = 0.3cm, 
                  labelposition = bottomline,
                  numbers       = left,
                  numbersep     = -0.2cm,
                  xleftmargin   = 0.8cm,
                  xrightmargin  = 0.8cm,
                ]
    import java.util.*;
    
    public class While extends Statement {
        BoolExpr        mCond;
        List<Statement> mStmntList;
        
        public While(BoolExpr cond, List<Statement> stmntList) {
            mCond      = cond;
            mStmntList = stmntList;
        }
        public void execute() {
            while (mCond.eval()) {
                for (Statement stmnt: mStmntList) {
                    stmnt.execute();
                }
            }
        }
    }
\end{Verbatim}
\vspace*{-0.3cm}
\caption{Die Klasse \texttt{While}.}
\label{fig:While.java}
\end{figure}




Von den \"ubrigen Klassen zur Darstellung von Befehlen diskutieren wir noch die Klasse
\texttt{While}, die in Abbildung \ref{fig:While.java} gezeigt wird.
Diese Klasse stellt einen Befehl der Form
\\[0.2cm]
\hspace*{1.3cm}
$\mathtt{while}\;(b)\; \{ \;\textsl{stmnts}\; \}$
\\[0.2cm]
dar, wobei $b$ ein Boole'scher Ausdruck ist, w\"ahrend \textsl{stmnts} eine Liste von Befehlen 
ist.  Der Boole'sche Ausdruck wird in der Member-Variablen \texttt{mCond} gespeichert, die Liste von
Befehlen findet sich in der Member-Variablen \texttt{mStmntList}.  Zur Auswertung eines solchen
Befehls f\"uhren wir solange alle Befehle in der Liste \texttt{mStmntList} aus, wie die Auswertung des
Boole'schen Ausdrucks $b$ den Wert \texttt{true} ergibt.


\begin{figure}[!ht]
\centering
\begin{Verbatim}[ frame         = lines, 
                  framesep      = 0.3cm, 
                  labelposition = bottomline,
                  numbers       = left,
                  numbersep     = -0.2cm,
                  xleftmargin   = 0.8cm,
                  xrightmargin  = 0.8cm,
                ]
    public abstract class BoolExpr {
        public abstract Boolean eval();
    }
    public class Equal extends BoolExpr {
        Expr mLhs;
        Expr mRhs;
        
        public Equal(Expr lhs, Expr rhs) {
            mLhs = lhs;
            mRhs = rhs;
        }
        public Boolean eval() {
            return mLhs.eval() == mRhs.eval();
        }
    }        
\end{Verbatim}
\vspace*{-0.3cm}
\caption{Die Klassen \texttt{BoolExpr} und \texttt{Equal}}
\label{fig:BoolExpr.java}
\end{figure}

\pagebreak
Die abstrakte Klasse \texttt{BoolExpr} dient zur Darstellung Boole'scher Ausdr\"ucke.  In unserem Fall
sind das Ausdr\"ucke der Form
\\[0.2cm]
\hspace*{1.3cm}
$l \;\texttt{==}\; r$ \quad und \quad $l < r$,
\\[0.2cm]
wobei Gleichungen durch die Klasse \texttt{Equal} dargestellt werden, w\"ahrend Ungleichungen durch die
Klasse \texttt{LessThan} dargestelt werden, die beide von der Klasse \texttt{BoolExpr} abgeleitet
sind.  Abbildung \ref{fig:BoolExpr.java} zeigt die Klassen \texttt{BoolExpr} und \texttt{Equal}.
Die Klasse \texttt{BoolExpr} hat die beiden Member-Variablen \texttt{mLhs} und
\texttt{mRhs} und repr\"asentiert die Gleichung
\\[0.2cm]
\hspace*{1.3cm}
\texttt{mLhs == mRhs}.
\\[0.2cm]
Um diese Gleichung auszuwerten, werden rekursiv die linke und die rechte Seite der
Gleichung, die in \texttt{mLhs} und \texttt{mRhs} gespeichert sind, ausgewertet.
Anschlie{\ss}end wird das Ergebnis dieser Auswertung zur\"uck gegeben.
Die Klasse \texttt{LessThan} ist analog zur Klasse \texttt{Equal} aufgebaut und wird daher nicht
gezeigt.

\begin{figure}[!ht]
\centering
\begin{Verbatim}[ frame         = lines, 
                  framesep      = 0.3cm, 
                  labelposition = bottomline,
                  numbers       = left,
                  numbersep     = -0.2cm,
                  xleftmargin   = 0.0cm,
                  xrightmargin  = 0.0cm,
                ]
    import java.util.*;
    
    public abstract class Expr {
        public static Map<String, Double> sValueTable = new HashMap<String, Double>();
        
        public abstract Double eval();
    }
\end{Verbatim}
\vspace*{-0.3cm}
\caption{Die abstrakte Klasse \texttt{Expr}.}
\label{fig:Expr.java}
\end{figure}

 
Schlie{\ss}lich haben wir noch die Klassen, die zur Repr\"asentation von arithmetischen Ausdr\"ucken
ben\"otigt werden.   Diese Klassen werden alle von der abstrakten Klasse \texttt{Expr} abgeleitet, die
in Abbildung \ref{fig:Expr.java} gezeigt ist.  
\begin{enumerate}
\item Die Klasse \texttt{Expr} definiert die statische Variable \texttt{sValueTable}.  Diese Variable
      beinhaltet eine Hash\-tabelle, in der f\"ur jede Variable, der ein Wert zugewiesen wurde,
      der aktuelle Wert dieser Variablen gespeichert ist.
\item Weiter deklariert die Klasse die abstrakte Methode $\texttt{eval}()$, mit der ein
      Ausdruck ausgewertet 
      werden kann.
\end{enumerate}
Von der Klasse
\texttt{Expr} werden die Klassen \texttt{Sum}, \texttt{Difference}, \texttt{Product}, \texttt{Quotient},
\texttt{MyNumber} und \texttt{Variable} abgeleitet, die wir jetzt der Reihe nach diskutieren.
Abbildung \ref{fig:Sum2.java} zeigt die Klasse
\texttt{Sum}.  Da diese Klasse eine Summe der Form
\\[0.2cm]
\hspace*{1.3cm}
$l + r$ 
\\[0.2cm]
darstellt, hat diese Klasse zwei Member-Variablen \texttt{mLhs}  und \texttt{mRhs} um die beiden
Summanden $l$ und $r$ darzustellen.  Die Methode $\texttt{eval}$ wertet diese beiden
Member-Variablen getrennt aus und addiert das Ergebnis.  Die Klassen \texttt{Difference},
\texttt{Product} und \texttt{Quotient} sind analog zur Klasse \texttt{Sum} aufgebaut und werden
daher nicht weiter diskutiert.

\begin{figure}[!ht]
\centering
\begin{Verbatim}[ frame         = lines, 
                  framesep      = 0.3cm, 
                  labelposition = bottomline,
                  numbers       = left,
                  numbersep     = -0.2cm,
                  xleftmargin   = 0.8cm,
                  xrightmargin  = 0.8cm,
                ]
    public class Sum extends Expr {
        Expr mLhs;
        Expr mRhs;
        
        public Sum(Expr lhs, Expr rhs) {
            mLhs = lhs;
            mRhs = rhs;
        }
        public Double eval() {
            return mLhs.eval() + mRhs.eval();
        }
    }
\end{Verbatim}
\vspace*{-0.3cm}
\caption{Die Klasse \texttt{Sum}}
\label{fig:Sum2.java}
\end{figure}

\begin{figure}[!ht]
\centering
\begin{Verbatim}[ frame         = lines, 
                  framesep      = 0.3cm, 
                  labelposition = bottomline,
                  numbers       = left,
                  numbersep     = -0.2cm,
                  xleftmargin   = 0.8cm,
                  xrightmargin  = 0.8cm,
                ]
    public class Variable extends Expr {
        String mName;
        
        public Variable(String name) {
            mName = name;
        }    
        public Double eval() {
            return sValueTable.get(mName);
        }
    }
\end{Verbatim}
\vspace*{-0.3cm}
\caption{Die Klasse \texttt{Variable}.}
\label{fig:Variable2.java}
\end{figure}
\vspace*{\fill} \pagebreak

Abbildung \ref{fig:Variable2.java} zeigt die Implementierung der Klasse \texttt{Variable}.  Die
Methode $\texttt{eval}()$ wertet eine Variable dadurch aus, dass sie unter dem Namen der Variablen
in der Hashtabelle \texttt{sValueTable} den zugeordneten Wert nachschl\"agt.

\begin{figure}[!ht]
\centering
\begin{Verbatim}[ frame         = lines, 
                  framesep      = 0.3cm, 
                  labelposition = bottomline,
                  numbers       = left,
                  numbersep     = -0.2cm,
                  xleftmargin   = 0.0cm,
                  xrightmargin  = 0.0cm,
                ]
    import org.antlr.v4.runtime.*;
    import java.io.FileInputStream;
    import java.io.InputStream;
    
    import java.util.List;
    import java.io.*;
    
    public class SLInterpreter {
        public static void main(String[] args) throws Exception {
            for (String file: args) {
                if (!file.equals("-")) {
                    parseFile(file);
                } else {
                    parseInteractive();
                }
            }
            if (args.length == 0) {
                parseInteractive();
            }
        }
        private static void parseFile(String fileName) throws Exception {
            try {
                FileInputStream fis = new FileInputStream(fileName);
                parseAndExecute(fis);
            } catch (IOException e) {
                System.err.println("File " + fileName + " could not be read.");
            }
        }
        private static void parseInteractive() throws Exception {
            Statement.isInteractive = true;
            Statement.prompt();
            while (true) {
                InputStream stream = InputReader.getStream();
                parseAndExecute(stream);
            }
        }
        private static void parseAndExecute(InputStream stream) 
            throws Exception 
        {
            ANTLRInputStream  input  = new ANTLRInputStream(stream);
            SimpleLexer       lexer  = new SimpleLexer(input);
            CommonTokenStream ts     = new CommonTokenStream(lexer);
            SimpleParser      parser = new SimpleParser(ts);
            parser.program();
        }
    }
\end{Verbatim}
\vspace*{-0.3cm}
\caption{Die Klasse \texttt{SLInterpreter}.}
\label{fig:SLInterpreter.java}
\end{figure}





Abbildung \ref{fig:SLInterpreter.java} zeigt das Treiber-Programm, das den von
\textsc{Antlr} erzeugten Parser einbindet.  Das Programm soll sp\"ater wahlweise in der Form
\\[0.2cm]
\hspace*{1.3cm}
\texttt{java} \texttt{SLInterpreter} $\textsl{file}_1$ $\cdots$ $\textsl{file}_n$
\\[0.2cm]
oder einfach als
\\[0.2cm]
\hspace*{1.3cm}
\texttt{java SLInterpreter}
\\[0.2cm]
aufgerufen werden.  Im ersten Fall  bezeichnet $\textsl{file}_i$ f\"ur $i=1,\cdots,n$
jeweils eine Datei, die ein auszuf\"uhrendes Programm enth\"alt.  Im zweiten Fall, oder
falls anstelle eines Dateinamens der String ``\texttt{-}'' als Argument \"ubergeben wird,
sollen die Befehle statt dessen interaktiv
eingegeben werden.  Die Methode $\textsl{parseFile}()$ behandelt dabei den Fall,
dass die Befehle aus einer Datei gelesen werden, w\"ahrend die Methode
$\texttt{parseInteractive}()$ f\"ur den Fall der interaktiven Verarbeitung zust\"andig ist.
Die in der Methode $\texttt{parseInteractive}()$ verwendete Klasse \texttt{InputReader} wird
dazu ben\"otigt, um von der Standardeingabe zu lesen und einen
\texttt{ANTLRInputStream} zu erzeugen.  Wir werden diese Klasse gleich genauer
diskutieren.  Sowohl die Methode $\texttt{parseFile}()$ als auch die Methode
$\texttt{parseInteractive}()$ dienen beide nur dazu, ein Objekt der Klasse
\texttt{ANTLRInputStream} zu erzeugen.  Im ersten Fall wird dieses Objekt aus der zu
lesenden Datei erzeugt, im zweiten Fall benutzen wir dazu den noch zu diskutierenden \texttt{InputReader}.
In beiden F\"allen wird schlie{\ss}lich die Methode $\texttt{parseAndExecute}()$ aufgerufen,
deren Aufgabe es ist, die einzelnen Befehle zu erkennen und auszuf\"uhren.


\begin{figure}[!ht]
\centering
\begin{Verbatim}[ frame         = lines, 
                  framesep      = 0.3cm, 
                  labelposition = bottomline,
                  numbers       = left,
                  numbersep     = -0.2cm,
                  xleftmargin   = 0.0cm,
                  xrightmargin  = 0.0cm,
                ]
    grammar Simple;
    
    @header {
        import java.util.List;
        import java.util.ArrayList;
    }
    
    program 
        : (s = statement { $s.stmnt.execute(); Statement.prompt(); })+
        ;
    statement returns [Statement stmnt]
        @init { 
            List<Statement> stmnts = new ArrayList<Statement>(); 
        }
        : v = VAR ':=' e = expr ';'     
          { $stmnt = new Assignment(new Variable($v.text), $e.result); }
        | v = VAR ':=' 'read' '(' ')' ';'     
          { $stmnt = new Read(new Variable($v.text)); }
        | 'print' '(' r = expr ')' ';' 
          { $stmnt = new Print($r.result); }
        | 'if' '(' b = boolExpr ')' '{' 
              (l = statement { stmnts.add($l.stmnt); })*
          '}' 
          { $stmnt = new IfThen($b.result, stmnts); }
        | 'while' '(' b = boolExpr ')' '{' 
              (l = statement { stmnts.add($l.stmnt); })*
          '}' 
          { $stmnt = new While($b.result, stmnts); }
        ;
    boolExpr returns [BoolExpr result]
        : l = expr '==' r = expr { $result = new Equal(   $l.result, $r.result); } 
        | l = expr '<'  r = expr { $result = new LessThan($l.result, $r.result); }
        ;
    expr returns [Expr result]
        : e = expr '+' p = product { $result = new Sum(       $e.result, $p.result); }
        | e = expr '-' p = product { $result = new Difference($e.result, $p.result); }
        | p = product              { $result = $p.result;                            }
        ;
    product returns [Expr result]
        : p = product '*' f = factor { $result = new Product( $p.result, $f.result); }
        | p = product '/' f = factor { $result = new Quotient($p.result, $f.result); }
        | f = factor                 { $result = $f.result;                          }
        ;    
    factor returns [Expr result]
        : '(' expr ')' { $result = $expr.result;          }
        | v = VAR      { $result = new Variable($v.text); }
        | n = NUMBER   { $result = new MyNumber($n.text); }
        ;
\end{Verbatim}
\vspace*{-0.3cm} %\$
\caption{\textsc{Antlr}-Spezifikation der Grammatik.}
\label{fig:Simple.g}
\end{figure}

\vspace*{\fill} \pagebreak

\vspace*{\fill} \pagebreak

\noindent
Abbildung \ref{fig:Simple.g} zeigt die Implementierung des Parsers mit dem
Werkzeug \textsc{Antlr}.  Der Parser liest in Zeile 8 eine nicht-leere Folge von Befehlen,
die sofort nach dem Einlesen ausgef\"uhrt werden.  Die \"ubrigen Grammatik-Regeln erzeugen
jeweils einen abstrakten Syntax-Baum der erkannten Eingabe.  So liefert beispielsweise die
Regel f\"ur die syntaktische Variable \texttt{statement} als Ergebnis ein Objekt der
abstrakten Klasse \texttt{Statement} zur\"uck.  Wir diskutieren nur den Fall, dass es sich
bei dem Statement um eine \texttt{while}-Schleife handelt, denn die anderen F\"alle sind
analog.  Zun\"achst wird in Zeile 11 in der Variablen \texttt{stmnts} eine leere Liste  von
\texttt{Statement}s angelegt. 
Diese Liste enth\"alt sp\"ater alle Befehle, die im Rumpf der \texttt{while}-Schleife stehen.
Anschlie{\ss}end wird in Zeile 21 das Schl\"usselwort ``\texttt{while}'' zusammen mit der Bedingung erkannt.
Dann werden der Reihe nach alle Befehle, die sich im Rumpf der Schleife befinden, der
Liste \texttt{stmnts} hinzugef\"ugt.  Nach dem Lesen der schlie{\ss}enden geschweiften Klammer
wird als R\"uckgabewert ein Objekt der Klasse \texttt{While} erzeugt und zur\"uckgegeben.


\begin{figure}[!ht]
\centering
\begin{Verbatim}[ frame         = lines, 
                  framesep      = 0.3cm, 
                  firstnumber   = last,
                  labelposition = bottomline,
                  numbers       = left,
                  numbersep     = -0.2cm,
                  xleftmargin   = 0.0cm,
                  xrightmargin  = 0.0cm,
                ]
    VAR    : [a-zA-Z][a-zA-Z_0-9]*;
    NUMBER : '0'|[1-9][0-9]*;
    
    MULTI_COMMENT : '/*' .*? '*/' -> channel(HIDDEN);
    LINE_COMMENT  : '//' ~('\n')* -> channel(HIDDEN);
    WS            : [ \t\v\n\r]   -> channel(HIDDEN);
\end{Verbatim}
\vspace*{-0.3cm}
\caption{\textsc{Antlr}-Spezifikation der Token.}
\label{fig:Simple-2.g}
\end{figure}
Die Spezifikation der Token ist in Abbildung \ref{fig:Simple-2.g}
gezeigt.  Der Scanner unterscheidet im Wesentlichen zwischen Variablen und Zahlen.
Variablen beginnen mit einem gro{\ss}en oder kleinen Buchstaben, auf den dann zus\"atzlich Ziffern und der
Unterstrich folgen k\"onnen.  Folgen von Ziffern werden als Zahlen interpretiert.  Enth\"alt eine solche
Folge mehr als ein Zeichen, so darf die erste Ziffer nicht 0 sein.  Dar\"uber hinaus entfernt der
Scanner Whitespace und Kommentare.  Beachten Sie, dass Whitespace und Kommentare an den
\texttt{channel}  mit den Namen \texttt{HIDDEN} weitergereicht werden.  Dadurch werden sie vom
Parser ignoriert, stehen aber noch f\"ur Fehlermeldungen zur Verf\"ugung.  Au{\ss}erdem haben wir bei der
Spezifikation von mehrzeiligen Kommentaren die sogenannte \emph{non-greedy}-Version des Operators
``\texttt{*}'' benutzt.  Die non-greedy-Version des Operators ``\texttt{*}'' wird als
``\texttt{*?}'' geschrieben und matched sowenig wie m\"oglich.  Daher steht der regul\"are Ausdruck
\\[0.2cm]
\hspace*{1.3cm}
\texttt{'/*' .*? '*/'}
\\[0.2cm]
f\"ur einen String, der mit der Zeichenkette ``\texttt{/*}'', mit der Zeichenkette ``\texttt{*/}''
endet und au{\ss}erdem so kurz wie m\"oglich ist.  Dadurch werden in einer Zeile der Form
\\[0.2cm]
\hspace*{1.3cm}
\texttt{/* Hugo */ i := i + 1; /* Anton */}
\\[0.2cm] 
zwei getrennte Kommentare erkannt.


\begin{figure}[!ht]
\centering
\begin{Verbatim}[ frame         = lines, 
                  framesep      = 0.3cm, 
                  firstnumber   = 1,
                  labelposition = bottomline,
                  numbers       = left,
                  numbersep     = -0.2cm,
                  xleftmargin   = 0.0cm,
                  xrightmargin  = 0.0cm,
                ]
    public final class InputReader {
        private static BufferedReader br  = null;
        private static String         EOL = "\n";
    
        public static InputStream getStream() throws EOFException {
            if (br == null) {
                br = new BufferedReader(new InputStreamReader(System.in));
            }
            String input     = "";
            String line      = null;
            int    endlAdded = 0;
            try {
                while (true) {
                    // line is read and returned without termination character(s)
                    line = br.readLine();
                    // add line termination (Unix style '\n' by default)
                    input     += line + EOL;   
                    endlAdded += EOL.length();
                    if (line == null) {
                        throw new EOFException("EndOfFile");
                    } else if (line.length() == 0 && input.length() > endlAdded) {
                        byte[] byteArray = 
                        input.substring(0, input.length() - EOL.length()).getBytes();
                        return new ByteArrayInputStream(byteArray);
                    }
                }
            } catch (IOException e) {
                e.printStackTrace();
            }
            return null;
        }
    }
\end{Verbatim}
\vspace*{-0.3cm}
\caption{Die Klasse \texttt{InputReader}.}
\label{fig:InputReader.java}
\end{figure}

Abbildung \ref{fig:InputReader.java} zeigt die Klasse \texttt{InputReader}, die dann benutzt wird,
wenn der Interpreter interaktiv betrieben wird.  Hier m\"ussen wir uns zun\"achst \"uberlegen, woran der
Interpreter erkennen soll, dass der Benutzer ein Kommando eingegeben hat.  Ein Ansatz w\"are, dass der
Parser nach jedem Zeilenumbruch \"uberpr\"uft, ob der Benutzer ein vollst\"andiges Kommando eingegeben
hat.  Dieser Ansatz scheitert aber daran, dass manche Kommandos sich \"uber mehrere Zeilen
erstrecken.  Daher wartet der Interpreter darauf, dass nacheinander zwei Zeilenumbr\"uche eingegeben
werden.  Dann wird die bis dahin gelesene Eingabe in Zeile 23 in einem Feld von Bytes
zusammengefasst und als Objekt der Klasse \texttt{ByteArrayInputStream} zur\"uck gegeben.


 \exercise
\begin{enumerate}
\renewcommand{\labelenumi}{(\alph{enumi})}
\item Erweitern Sie den Interpreter so, dass auch der Operator ``\texttt{<=}'' unterst\"utzt wird.
\item Erweitern Sie den Interpreter um \texttt{for}-Schleifen.
\item Erweitern Sie den Interpreter um die logischen Operatoren
      ``\texttt{\&\&}'' f\"ur das logische \emph{Und}, ``\texttt{||}'' f\"ur das logische \emph{Oder}
      und ``\texttt{!}'' f\"ur die Negation an.  Dabei soll der Operator ``\texttt{!}'' am
      st\"arksten und der Operator ``\texttt{||}'' am schw\"achsten binden.
\item Erweitern Sie die Syntax der arithmetischen Ausdr\"ucke so, dass auch vordefinierte 
      mathematische Funktionen wie $\texttt{exp}()$ oder $\texttt{ln}()$ benutzt werden
      k\"onnen.

      \textbf{Hinweis}:  Wenn Sie das Paket \texttt{java.lang.reflect} benutzen,
      kommen Sie mit einer zus\"atzlichen Klasse aus und k\"onnen damit alle in
      \texttt{java.lang.Math} definierten Methoden implementieren.
\item Erweitern Sie den Interpreter so, dass auch benutzerdefinierte Funktionen m\"oglich
      werden. 
      
      \textbf{Hinweis}: Jetzt m\"ussen Sie zwischen lokalen und globalen Variablen
      unterscheiden.
      Daher reicht es nicht mehr, die Belegungen der Variablen in einer
      global definierten Hashtabelle zu verwalten. \eox
\item Erweitern Sie den Interpreter so, dass die Verwendung rationaler Zahlen unterst\"utzt wird.
      Das Ziel dieser Erweiterung ist eine Sprache, mit der Sie Rechnungen ohne Rundungsfehler durchf\"uhren k\"onnen.
\end{enumerate}
\renewcommand{\labelenumii}{\arabic{enumii}.}


%%% Local Variables: 
%%% mode: latex
%%% TeX-master: "formal-languages"
%%% End: 
