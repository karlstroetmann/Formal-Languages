\documentclass{article}
\usepackage{ngerman}
\usepackage[latin1]{inputenc}
\usepackage{a4wide}
\usepackage{epsfig}
\usepackage{amssymb}
\usepackage{fancyvrb}
\usepackage{alltt}
\usepackage{fleqn}
\usepackage{theorem}

\setlength{\mathindent}{1.3cm}


\newcommand{\exercise}{\vspace*{0.1cm}
\stepcounter{aufgabe}

\noindent
\textbf{Aufgabe \arabic{aufgabe}}: }

\newcommand{\proof}{\vspace*{0.1cm}

\noindent
\textbf{Beweis}: }

\newcommand{\remark}{\vspace*{0.1cm}

\noindent
\textbf{Bemerkung}: }


\newcommand{\solution}{\vspace*{0.1cm}

\noindent
\textbf{L\"osung}: }

\newcommand{\example}{\vspace*{0.2cm}

\noindent
\textbf{Beispiel}: \ }

\newcommand{\examples}{\vspace*{0.2cm}

\noindent
\textbf{Beispiele}: \ }

%\newcommand{\next}{\vspace*{0.2cm}}

\newcommand{\shft}[1]{$\pair(\textsl{shft},s_{#1})$}
\newcommand{\rdc}[1]{$\pair(\textsl{rdc},r_#1)$}
\newcommand{\accept}{\textsl{accept}}

\newcommand{\squoted}[1]{\mbox{``\texttt{#1}''}}
\newcommand{\quoted}[1]{\;\mbox{``\texttt{#1}''}\;}
\newcommand{\qote}[1]{``\texttt{#1}''}
\newcommand{\comp}[1]{\stackrel{#1}{\mapsto}}
\newcommand{\dx}{\,\textrm{d}}
\newcommand{\bruch}[2]{\frac{\displaystyle#1}{\displaystyle #2}}
\newcommand{\folge}[1]{\bigl(#1\bigr)_{n\in\mathbb{N}}}
\def\pair(#1,#2){\langle #1, #2 \rangle}
\def\trip(#1,#2,#3){\langle #1, #2, #3 \rangle}
\newcommand{\qed}{\hspace*{\fill} $\Box$}
\newcommand{\lmderiv}{\;\raisebox{-2.3mm}{$\stackrel{\mbox{\large$\Rightarrow$}}{\mbox{\scriptsize l}}$}\;}


{\theorembodyfont{\sf}
\newtheorem{Definition}{Definition}
\newtheorem{Satz}[Definition]{Satz}
\newtheorem{Lemma}[Definition]{Lemma}
\newtheorem{Korollar}[Definition]{Korollar}
}

\newcounter{aufgabe}


\title{Eindeutige kontextfreie Grammatiken zur Festlegung von Operator-Pr\"azedenzen}
\author{Karl Stroetmann}

\begin{document}
\maketitle

In den meisten Programmiersprachen gibt es eine gr\"o{\ss}ere Zahl von Operatoren, f\"ur die 
unterschiedliche Bindungsst\"arken festgelegt werden.  Wir nehmen an, dass im allgemeinen
Dall eine Menge \textsl{Operators} von Operatoren gegeben ist, die folgende Form hat:
\[ 
   \textsl{Operators} = 
   \bigl\{ \oplus_{i,j} \mid i \in \{1,\cdots,n\} \wedge j \in \{ 1, \cdots, n(i) \} \bigl\}
\]
Die Idee ist dabei, dass es $n$ verschieden Priorit\"ats-Stufen gibt:  F\"ur ein gegebenes 
$i \in \{1,\cdots,n\}$ gibt es $n(i)$ verschieden Operatoren, die in der der Menge
$\bigl\{ \oplus_{i,j} \mid j \in \{ 1, \cdots, n(i) \} \bigl\}$
zusammengefasst sind.  Alle diese  Operatoren $\oplus_{i,j}$ haben die Pr\"azedenz $i$.
Eine einfache Grammatik, welche die von diesen Operatoren erzeugte Sprache beschreibt,
besteht aus folgenden Regeln:
\begin{enumerate}
\item $\textsl{Expr} \rightarrow \textsl{Expr} \;\oplus_{i,j}\, \textsl{Expr}$

      f\"ur alle $i \in \{ 1, \cdots, n \}$ und alle $j \in \{ 1, \cdots, n(i) \}$.
\item $\textsl{Expr} \rightarrow \squoted{(} \;\textsl{Expr}\; \squoted{)}$
\item $\textsl{Expr} \rightarrow \textsl{Simple}$

      Hier steht \textsl{Simple} f\"ur eine syntaktische Variable, die beispielsweise
      Zahlen und Strings erzeugt.
\end{enumerate}
Diese Grammatik ist allerdings mehrdeutig und zum Parsen daher nicht geeignet.

\begin{figure}[htbp]
  \begin{center}    
  \framebox{
  \framebox{
  \begin{minipage}[t]{9cm}
  \begin{eqnarray*}
  \textsl{Expr}        & \rightarrow & \;\textsl{Product}\;\;\textsl{ExprRest}            \\[0.2cm]
  \textsl{ExprRest}    & \rightarrow & \quoted{+} \textsl{Product}\;\;\textsl{ExprRest}   \\
                       & \mid        & \quoted{-} \textsl{Product}\;\;\textsl{ExprRest}   \\
                       & \mid        & \;\varepsilon                                      \\[0.2cm]
  \textsl{Product}     & \rightarrow & \;\textsl{Factor}\;\;\textsl{ProductRest}          \\[0.2cm]
  \textsl{ProductRest} & \rightarrow & \quoted{*} \textsl{Factor}\;\;\textsl{ProductRest} \\
                       & \mid        & \quoted{/} \textsl{Factor}\;\;\textsl{ProductRest} \\
                       & \mid        & \;\varepsilon                                      \\[0.2cm]
  \textsl{Factor}      & \rightarrow & \quoted{(} \textsl{Expr} \quoted{)}                \\
                       & \mid        & \;\textsc{Number} 
  \end{eqnarray*}
  \vspace*{-0.5cm}
  \end{minipage}}}
  \end{center}
  \caption{Grammatik f\"ur arithmetische Ausdr\"ucke ohne Links-Rekursion.}
  \label{fig:Expr2}
\end{figure}



\end{document}

