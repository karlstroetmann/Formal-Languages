
\subsection{Implementing a Recursive Decent Parser that Uses an \textsc{EBNF} Grammar}
The previous two solutions to parse an arithmetical expressions were not completely
satifying:  The reason is that we did not really fix the problem but rather cured the
symptoms.  The real problem is that context free grammars are not that convenient to
describe programming languages.  Let us extend the power of context free languages
slightly by admitting regular expression on the right hand side of grammar rules.  
These new type of grammars are known as
\href{http://en.wikipedia.org/wiki/Extended_Backus_Naur_Form}{\emph{extended Backus Naur form}}
grammars, which 
is abbreviated as \textsc{Ebnf} grammars.  An \textsc{Ebnf} grammar admits the operators
\squoted{*}, \squoted{?}, \squoted{+}, and \squoted{|} on the right hand side of a grammar
rule.  The meaning of these operators is the same as when these operators are used in 
regular expressions.

It can be shown that the languages described by \textsc{Ebnf} grammars are still context
free languages.  Therefore, these operators do not change the expressive power of context-free
grammars. 
However, it is often much more \underline{convenient} to describe a language using an \textsc{Ebnf}
grammar rather than using a context free grammar.  Figure \ref{fig:arith-expr-ebnf}
displays an \textsc{Ebnf} grammar for arithmetical expressions.  We have extended this
grammar to allow for the exponentiation operator \squoted{**}.  In order to support this
operator, we had to introduce a new syntactical variable, which we called \textsl{Base}.
In an arithmetical expression, \textsl{Base} serves as the base of a power.  The
exponent can be an arbitrary \textsl{Factor}.  This way, an expression of the form
\\[0.2cm]
\hspace*{1.3cm}
\texttt{2 ** 3 ** 4}  \quad is parsed as \quad
\texttt{2 ** (3 ** 4)}
\\[0.2cm]
and therefore the operator \squoted{**} is right associative.  This is also the convention used in
mathematics. 
Furthermore, we have added the function symbols \squoted{exp} and \squoted{ln} to be able to
support the exponential function and the natural logarithm.  The grammar also supports
variables.  The reason is that we want to implement a program for \emph{symbolic differentiation}:
We want to implement a function that takes a string respresenting an 
arithmetical expression and then does the following:
\begin{enumerate}
\item In the first step, the string is translated into an abstract syntax tree.
\item In the second step, this tree is differentiated symbolically with respect to
      a given variable.
\end{enumerate}
For example, given the string
\\[0.2cm]
\hspace*{1.3cm}
\texttt{x * exp(x)}
\\[0.2cm]
the program is supposed to compute the answer
\\[0.2cm]
\hspace*{1.3cm}
\texttt{1 * exp(x) + x * exp(x)},
\\[0.2cm]
since we have
\\[0.2cm]
\hspace*{1.3cm}
$\frac{d\;}{dx} \bigl(x \cdot \exp(x)\bigr) = 1 \cdot \exp(x) + x \cdot \exp(x)$
\\[0.2cm]
Obviously, the grammar in Figure \ref{fig:arith-expr-ebnf}  is
more concise than the context free grammar discussed at the beginning of this chapter.
For example, the first rule clearly expresses that an arithmetical expression is a list of
products that are separated by the operators \squoted{+} and \squoted{-}.

\begin{figure}[htbp]
  \begin{center}    
  \framebox{
  \framebox{
  \begin{minipage}[t]{9cm}

  \begin{eqnarray*}
  \textsl{expr}    & \rightarrow & \;\textsl{product}\;\; \bigl((\squoted{+}\;|\;\squoted{-})\;\; \textsl{product}\bigr)^* \\[0.2cm]
  \textsl{product} & \rightarrow & \;\textsl{factor}\;\; \bigl((\squoted{*}\;|\;\squoted{/})\;\; \textsl{factor} \bigr)^* \\[0.2cm]
  \textsl{factor}  & \rightarrow & \;\textsl{base}\;\; (\squoted{**}\;\; \textsl{factor})?                                 \\
  \textsl{base}    & \rightarrow & \quoted{(} \textsl{expr} \quoted{)}                  \\
                   & \mid        & \quoted{exp} \quoted{(} \textsl{expr} \quoted{)}                  \\
                   & \mid        & \quoted{ln} \quoted{(} \textsl{expr} \quoted{)}                  \\
                   & \mid        & \;\textsc{Number}                                    \\
                   & \mid        & \;\textsc{Variable}                                    
  \end{eqnarray*}
  \vspace*{-0.5cm}

  \end{minipage} \hspace*{1.cm}}}
  \end{center}
  \caption{\textsc{Ebnf} grammar for arithmetical expressions.}
  \label{fig:arith-expr-ebnf}
\end{figure}

\noindent
Figure \ref{fig:differentiate.stlx} shows a parser that implements this grammar.
\begin{enumerate}
\item The function \texttt{parseArithExpr} recognizes a \texttt{product} in line 2. 
      The value of this \texttt{product} is stored in the variable 
      \texttt{result} together with the list \texttt{rl} of those tokens that have not been consumed
      yet.  If the list \texttt{rl} is not empty and the first token in this
      list is either the operator \squoted{+} or the operator \squoted{-},
      then the function \texttt{parseArithExpr} tries to recognize more products.
      These are added to or subtracted from the \texttt{result} computed so far in
      line 7 or 8.  If there are no more products to be parsed, the \texttt{while} loop 
      terminates and the function returns the \texttt{result} together with the remaining
      tokens.
\item The function \texttt{parseProduct} recognizes a \texttt{factor} in line 14. 
      The value of this \texttt{factor} is stored in the variable 
      \texttt{result} together with the list \texttt{rl} of those tokens that have not been consumed
      yet.  If the list \texttt{rl} is not empty and the first token in this
      list is either the operator \squoted{*} or the operator \squoted{/},
      then the function \texttt{parseProduct} tries to recognize more factors.
      The \texttt{result} computed so far is multiplied with or divided by these factors in
      line 19 or 20.  If there are no more products to be parsed, the \texttt{while} loop 
      terminates and the function returns the \texttt{result} together with the list
      \texttt{rl} of tokens that have not been consumed.
\item The function \texttt{parseFactor} recognizes a \texttt{factor}.  
      In any case, a factor starts with a base.  Optionally, a factor can be a power.
      This is the case if the base is followed by the exponentiation operator \squoted{**}.
\item The function \texttt{parseBase} recognizes a call of the exponential function, a
      call of the natural logarithm, a parenthesized expression, a number, or a variable.
      Fortunately, the first token of the token list tells us which case we have.
\end{enumerate}

\begin{figure}[!ht]
\centering
\begin{Verbatim}[ frame         = lines, 
                  framesep      = 0.3cm, 
                  firstnumber   = 1,
                  labelposition = bottomline,
                  numbers       = left,
                  numbersep     = -0.2cm,
                  xleftmargin   = 0.8cm,
                  xrightmargin  = 0.8cm,
                ]
    parseArithExpr := procedure(tl) {
        [result, rl] := parseProduct(tl);
        while (#rl > 1 && rl[1] in ["+", "-"]) {
            op := rl[1];
            [arg, rl] := parseProduct(rl[2..]);
            match (op) {
                case "+": result := result + arg;
                case "-": result := result - arg;
            } 
        }
        return [result, rl];
    };
    parseProduct := procedure(tl) {
        [result, rl] := parseFactor(tl);
        while (#rl > 1 && rl[1] in ["*", "/"]) {
            op := rl[1];
            [arg, rl] := parseFactor(rl[2..]);
            match (op) {
                case "*": result := result * arg;
                case "/": result := result / arg;
            } 
        }
        return [result, rl];
    };
    parseFactor := procedure(tl) {
        [atom, rl] := parseBase(tl);
        match (rl) {
            case [ "**" | ql ]: [factor, rl] := parseFactor(ql);
                                return [atom ** factor, rl];
            default:            return [atom, rl];
        }
    };    
    parseBase := procedure(tl) {
        match (tl) {
            case [ "exp", "(" | rl ]: [expr, ql] := parseArithExpr(rl);
                                      assert(ql[1] == ")", "Parse Error");
                                      return [ Exp(expr), ql[2..]];
            case [ "ln", "(" | rl ]:  [expr, ql] := parseArithExpr(rl);
                                      assert(ql[1] == ")", "Parse Error");
                                      return [ Ln(expr), ql[2..]];
            case ["(" | rl ]:         [expr, ql] := parseArithExpr(rl);
                                      assert(ql[1] == ")", "Parse Error");
                                      return [expr, ql[2..]];
            case [ Number(n) | rl ]:  return [Number(n), rl];
            case [ Var(v) | rl ]:     return [Var(v), rl];
            default:                  abort("parse error in parseBase($tl$)");
        }
    };
\end{Verbatim}
\vspace*{-0.3cm}
\caption{A parser for the grammar in Figure \ref{fig:arith-expr-ebnf}.}
\label{fig:differentiate.stlx}
\end{figure}

The program in Figure \ref{fig:differentiate.stlx} generates an abstract syntax tree.
This syntax tree is represented as a term in \textsc{SetlX}.  Note that in \textsc{SetlX}
it is possible to use operators as functors.  For example, if we have the expression
\\[0.2cm]
\hspace*{1.3cm}
$s + t$
\\[0.2cm]
and at least one of the arguments $s$ or $t$ is a term, then $s + t$ is a term, too.
This enables us to write
\\[0.2cm]
\hspace*{1.3cm}
\texttt{result := result + arg;}
\\[0.2cm]
in line 7 of Figure \ref{fig:differentiate.stlx}.  Finally, Figure \ref{fig:diff.stlx}
shows the implementation of the function \texttt{diff} that can be used for symbolic
differentiation.  The argument $t$ of this function is supposed to be a term, the second
argument $x$ is interpreted as the name of a variable.  For example, line 7 and 8 of Figure
\ref{fig:diff.stlx} implement the product rule.  We have
\\[0.2cm]
\hspace*{1.3cm}
$\ds\frac{d\;}{dx}(u \cdot v) = \frac{d\,u}{dx} \cdot v + u \cdot \frac{d\,v}{dx}$. 
\\[0.2cm]
The right hand side of this equation is returned in line 8: \texttt{t1} correspond to $u$
and \texttt{t2} corresponds to $v$.  The other rules for differentiation are implemented
in a similar way.

\begin{figure}[!ht]
\centering
\begin{Verbatim}[ frame         = lines, 
                  framesep      = 0.3cm, 
                  firstnumber   = 1,
                  labelposition = bottomline,
                  numbers       = left,
                  numbersep     = -0.2cm,
                  xleftmargin   = 0.8cm,
                  xrightmargin  = 0.8cm,
                ]
    diff := procedure(t, x) {
        match (t) {
            case t1 + t2 :
                 return diff(t1, x) + diff(t2, x);
            case t1 - t2 :
                 return diff(t1, x) - diff(t2, x);
            case t1 * t2 :
                 return diff(t1, x) * t2 + t1 * diff(t2, x);
            case t1 / t2 :
                 return ( diff(t1, x) * t2 - t1 * diff(t2, x) ) / (t2 * t2);
            case f ** Number(n): 
                return Number(n) * diff(f, x) * f ** Number(n-1);
            case f ** g :
                 return diff( Exp(g * Ln(f)), x);
            case Ln(a) :
                 return diff(a, x) / a;
            case Exp(a) : 
                 return diff(a, x) * Exp(a);
            case Var(x) : // x is defined above as second argument to diff
                 return 1;
            case Var(y) : // y is undefined, therefore matches any other variable
                 return 0;
            case Number(n):
                 return 0;    
            default:
                abort("error in diff($t$, $x$)");
        }
    };
\end{Verbatim}
\vspace*{-0.3cm}
\caption{A function for symbolic differentiation.}
\label{fig:diff.stlx}
\end{figure}


\paragraph{Historical Notes} The language \textsc{Algol} \cite{backus:1959,naur:1960} was the first
programming language with a syntax that was based on a context-free grammar.  

%%% Local Variables: 
%%% mode: latex
%%% TeX-master: "formal-languages.tex"
%%% End: 
