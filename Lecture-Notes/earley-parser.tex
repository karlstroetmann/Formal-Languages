\chapter{Earley Parser}
In this Chapter we will discuss an efficient algorithm that takes two inputs:
\begin{enumerate}
\item A context-free grammar $G = \langle V, \Sigma, R, S \rangle$ and
\item a string $s \in \Sigma^*$.
\end{enumerate}
The algorithm decides, whether $s \in L(G)$ holds, i.e.~it checks, whether the string is a
member of the language generated by th \textsc{Cfg} $G$.

The algorithm that is presented next has been published in 1970 by Jay Earley \cite{earley:70}.
There is another algorithm solving the same problem, namely the
\href{https://en.wikipedia.org/wiki/CYK_algorithm#:~:text=In%20computer%20science%2C%20the%20Cocke,Tadao%20Kasami%2C%20and%20Jacob%20T.}{Cocke-Younger-Kasami algorithm},
\index{Cocke-Younger-Kasami-Algorithmus}
  which is also known as the
\blue{\textsc{Cyk} algorithm}.\index{\textsc{Cyk}-Algorithmus}
It has been discovered independently by John Cocke
\cite{cocke:1970}, Daniel H.~Younger \cite{younger:1967}, and Tadao Kasami \cite{kasami:1965}.
The \textsc{Cyk} algorithm can only be used when the grammar has a special form, namely it has to be in
\href{https://en.wikipedia.org/wiki/Chomsky_normal_form}{Chomsky normal form}.
As it is quite tedious to transform a given grammar into Chomsky normal form, this algorithm is not used in
practical applications.
In contrast, the Earley algorithm works for arbitrary context-free grammars and hence is used in practical applications.
In the general case, the Earley algorithm has a complexity of $\mathcal{O}(n^3)$ where $n$ is the length of the
string that is to be parsed.  However, if the grammar is not ambiguous, the complexity is only
$\mathcal{O}(n^2)$.  Skillful implementations of Earley's algorithm even achieve a linear runtime for many
practically relevant grammars.   
For example, Earley's algorithm has a linear complexity for both  $LL(k)$ grammars and also for $LR(1)$
grammars.  We will discuss $LR(1)$ grammars in a later Chapter.  On the contrary, the \textsc{Cyk} algorithm
always has the complexity $\mathcal{O}(n^3)$, which is prohibitive for practical applications.

This chapter is structured as follows:
\begin{enumerate}[(a)]
\item First, we sketch the theory of Earley's algorithm.
\item Next, we show how this algorithm can be implemented in \textsl{Python}.
%\item Anschlie�end beweisen wir die Korrektheit und Vollst\"andigkeit des Algorithmus.
%\item Zum Abschluss des Kapitels untersuchen wir die Komplexit\"at.
\end{enumerate}

\section{The Earley Algorithm}
The central notion that is needed to understand Earley's algorithm is the notion of an
\blue{Earley object}\index{Earley object}.  This notion is defined below.

\begin{Definition}[Earley Object]
  Assume that  $G = \langle V, \Sigma, R, s \rangle$ is a context-free grammar and
  $w = x_1x_2 \cdots x_n \in \Sigma^*$ is a string of length $n$.  A pair of the form
  \\[0.2cm]
  \hspace*{1.3cm}
  $\langle A \rightarrow \alpha \bullet \beta, k \rangle$
  \\[0.2cm]
  is called an \blue{Earley object} if and only if
  \begin{enumerate}[(a)]
  \item $(A \rightarrow \alpha \beta) \in R$ \quad and
  \item $k \in \{0,1,\cdots,n\}$. \eox
  \end{enumerate}
\end{Definition}

\noindent
\textbf{Explanation}: 
An Earley describes a possible state of a parser.
If a parser has to parse a string of the form $x_1 \cdots x_n$, then this parser will maintain
$n+1$ sets of Earley objects.  These sets are denoted as
\\[0.2cm]
\hspace*{1.3cm}
$Q_0, Q_1, \cdots, Q_n$.
\\[0.2cm]
If $i \in \{0,1, \cdots, n\}$, then the state $Q_i$ contains those Earley objects that describe the state of
the parser after it has parsed the tokens $x_1$, $\cdots$, $x_i$.  
The interpretation of 
\\[0.2cm]
\hspace*{1.3cm}
$\langle a \rightarrow \beta \bullet \gamma, k \rangle \in Q_j$ \quad where  $j \geq k$
\\[0.2cm]
is as follows:
\begin{enumerate}
\item The parser tries to use the grammar rule $a \rightarrow \beta \gamma$ to parse the variable $a$ at the
      beginning of the substring $x_{k+1} \cdots x_n$.
\item The parser has already parsed $\beta$ in the substring  $x_{k+1} \cdots x_j$, we have
      \\[0.2cm]
      \hspace*{1.3cm}
      $\beta \Rightarrow^* x_{k+1} \cdots x_j$.
\item Hence, in order to parse the variable $a$ the parser only needs to recognize $\gamma$ at the beginning of
      the substring $x_{j+1} \cdots x_n$.
\end{enumerate}
Earley's  algorithm manages the sets $Q_0$, $Q_1$, $\cdots$, $Q_n$  of Earley objects.  The set $Q_j$ contains
those Earley objects that contain all those states the parser could be in when it has read the prefix 
$x_1 \cdots x_j$.

At the beginning of the algorithm we add a new start variable  $\widehat{s}$ to the grammar.  Furthermore the
rule $\widehat{s} \rightarrow s$ is added to the grammar.  Hence, the grammar $G = \langle V, \Sigma, R, s \rangle$
is transformed into the \blue{augmented grammar} 
\\[0.2cm]
\hspace*{1.3cm}
$\widehat{G} = \langle V \cup \{\widehat{s}\}, \Sigma, R \cup \{ \widehat{s} \rightarrow s \}, \widehat{s} \,\rangle$. 
\\[0.2cm]
Next, the set $Q_0$ is defined as
\\[0.2cm]
\hspace*{1.3cm}
$Q_0 := \bigl\{ \pair(\widehat{S} \rightarrow \bullet S, 0) \bigr\}$.
\\[0.2cm]
The reason is that the  parser should recognize the start symbol $s$ an the beginning of the string $x_1 \cdots
x_n$.
The remaining sets  $Q_j$ are initially empty for $j=1,\cdots,n$ leer.  These sets are extended by the
following three operations: 
\begin{enumerate}
\item \emph{Reading}

      If the set $Q_j$ contains an Earley object of the form 
      $\pair(a \rightarrow \beta \bullet T \gamma, k)$ and $T$ is a 
      terminal, then the parser tries to parse the right hand side of the grammar rule
      $a \rightarrow \beta T \gamma$ and, after reading $x_{k+1} \cdots x_j$ it has already recognized $\beta$.
      If this $\beta$ is now followed by the token $T$, then in order to recognize $a$ with the grammar rule
      $a \rightarrow \beta T \gamma$, the parser only has to recognize $\gamma$ in the substring
      $x_{j+2} \cdots x_n$.  Hence, in this case we add the Earley object
      \\[0.2cm]
      \hspace*{1.3cm}
      $\pair(a \rightarrow \beta T \bullet \gamma, k)$
      \\[0.2cm]
      to the set $Q_{j+1}$:
      \\[0.2cm]
      \hspace*{1.3cm}
      $\pair(a \rightarrow \beta \bullet T \gamma, k) \in Q_j \wedge x_{j+1} = T
       \;\Rightarrow\;
       Q_{j+1} := Q_{j+1} \cup \bigl\{ \pair(a \rightarrow \beta T \bullet \gamma, k) \bigr\}$.
\item \emph{Prediction}

      If the set $Q_j$ contains an Earley object of the form $\pair(a \rightarrow \beta \bullet c \delta, k)$
      such that $c$ is a variable, then the parser tries to recognize the substring $C\delta$ after having
      parsed the substring $x_{k+1} \cdots x_j$ as $\beta$.  Hence the parser now has to recognize the variable
      $c$.  Therefore, for every rule of the form $c \rightarrow \gamma$ that is contained in the grammar $G$
      the Earley object 
      $\pair(c \rightarrow \bullet \gamma, j)$ is added to the set $Q_j$:
      \\[0.2cm]
      \hspace*{1.3cm}
      $\pair(a \rightarrow \beta \bullet c \delta, k) \in Q_j 
       \wedge (c \rightarrow \gamma) \in R 
       \;\Rightarrow\;
       Q_j := Q_j \cup\bigl\{ \pair(c \rightarrow \bullet\gamma, j)\bigr\}$.
\item \emph{Completion}

      If the set $Q_i$ contains an Earley object of the form $\pair(c \rightarrow \gamma \bullet, j)$
      and the set $Q_j$ contains an Earley object of the form 
      $\pair(a \rightarrow \beta \bullet c \delta,k)$, then the parser has tried to parse the variable $c$
      after it had read the substring $x_1\cdots x_{j}$ and after reading the substring $x_{j+1}\cdots x_i$
      it has recognized the variable $c$.
      Therefore, the Earley object
      $\pair(a \rightarrow \beta c \bullet \delta,k)$
      is now added to the set $Q_i$:
      \\[0.2cm]
      \hspace*{1.3cm}
      $\pair(c \rightarrow \gamma \bullet, j) \in Q_i \wedge
       \pair(a \rightarrow \beta \bullet c \delta,k) \in Q_j \;\Rightarrow\;
       Q_i := Q_i \cup \bigl\{ \pair(a \rightarrow \beta c \bullet \delta,k) \bigr\}
      $.
\end{enumerate}
When trying to parse the string $w = x_1 \cdots x_n$ with the grammar
$G = \langle V, \Sigma, R, s \rangle$ Earley's algorithm works as follows:
\begin{enumerate}
\item The sets  $Q_i$ are initiallized as follows:
      \\[0.2cm]
      \hspace*{1.3cm}
      $Q_0 := \bigl\{ \pair(\widehat{s} \rightarrow \bullet s, 0) \bigr\}$,
      \\[0.2cm]
      \hspace*{1.3cm}
      $Q_i := \bigl\{ \bigr\}$ \quad for $i=1,\cdots,n$.
\item After that we iterate from $i=0$ to $i=n$ and perform the following operations:
      \begin{enumerate}[(a)]
      \item The set $Q_i$ is enlarged using \blue{completion} until we find no further
            Earley objects.
      \item Next we use prediction to enlarge the set $Q_i$.
            Again, this operation is performed until no new Earley objects are found.
      \item If $i < n$, we read the next token and use it to initiallize $Q_{i+1}$.
      \end{enumerate}
      If the grammar $G$ has  $\varepsilon$-rules, i.e.~if it has rules with empty right hand side of the form
      \\[0.2cm]
      \hspace*{1.3cm}
      $C \rightarrow \lambda$,
      \\[0.2cm]
      then it might happen that after applying prediction in the set $Q_i$ we can apply immediately apply
      completion in $Q_i$.  In this case prediction and completion have to be iterated until no further Earley
      objects are generated for $Q_i$.
\item If after termination of the algorithm the set  $Q_n$ contains the Earley object
      $\pair(\widehat{s} \rightarrow s \bullet,0)$, then parsing is successfull and we have shown that
      the string $w = x_1 \cdots x_n$ is an element of $L(G)$.
\end{enumerate}
  
\example
Abbildung \ref{fig:expr-small} zeigt eine vereinfachte Grammatik f\"ur arithmetische
Ausdr\"ucke, die nur aus den Zahlen ``1'', ``2'' und ``3'' und den beiden Operator-Symbolen
``\texttt{+}'' und ``\texttt{*}'' aufgebaut ist.  Die Menge $T$ der Terminale dieser
Grammatik ist also durch
\\[0.2cm]
\hspace*{1.3cm}
 $T = \{ \quoted{1}, \quoted{2}, \quoted{3}, \quoted{+}, \quoted{*} \}$
\\[0.2cm]
gegeben.
Wir zeigen, wie sich der String
``\texttt{1+2*3}'' mit dieser Grammatik und dem Algorithmus von Earley parsen l\"asst.
In der folgenden Darstellung werden wir die syntaktische Variable \texttt{expr} mit
dem Buchstaben $E$ abk\"urzen, f\"ur \texttt{prod} schreiben wir $P$ und f\"ur \texttt{fact} verwenden wir die
Abk\"urzung $F$.


\begin{figure}[!ht]
\centering
\begin{Verbatim}[ frame         = lines, 
                  framesep      = 0.3cm, 
                  labelposition = bottomline,
                  numbers       = left,
                  numbersep     = -0.2cm,
                  xleftmargin   = 0.8cm,
                  xrightmargin  = 0.8cm,
                ]
    expr : expr '+' prod
         | prod
         ;
    
    prod : prod '*' fact
         | fact
         ;
    
    fact : '1'
         | '2'
         | '3'
         ;
\end{Verbatim}
\vspace*{-0.3cm}
  \caption{Eine vereinfachte Grammatik f\"ur arithmetische Ausdr\"ucke.}
  \label{fig:expr-small} 
\end{figure}


\begin{enumerate}
\item Wir initialisieren $Q_0$ als
      \\[0.2cm]
      \hspace*{1.3cm}
      $Q_0 = \{ \pair(\widehat{S} \rightarrow \bullet\, E, 0) \}$. 
      \\[0.2cm]
      Die Mengen $Q_1$, $Q_2$, $Q_3$, $Q_4$ und $Q_5$ sind zun\"achst alle leer.
      Wenden wir die Vervollst\"andigungs-Operation auf $Q_0$ an, so finden wir keine neuen
      Earley-Objekte.

      Anschlie{\ss}end wenden wir die Vorhersage-Operation auf das Earley-Objekt 
      $\pair(\widehat{S} \rightarrow \bullet\, E, 0)$ an.  Dadurch werden der Menge $Q_0$ 
      zun\"achst die beiden Earley-Objekte 
      \\[0.2cm]
      \hspace*{1.3cm}
      $\pair(E \rightarrow \bullet\; E \squoted{+} P, 0)$ 
      \quad und \quad
      $\pair(E \rightarrow \bullet\; P, 0)$ 
      \\[0.2cm]
      hinzugef\"ugt.  Auf das Earley-Objekt $\pair(E \rightarrow \bullet\, P, 0)$ 
      k\"onnen wir die Vorhersage-Operation ein weiteres Mal anwenden und erhalten dann die beiden
      neuen Earley-Objekte
      \\[0.2cm]
      \hspace*{1.3cm}
      $\pair(P \rightarrow \bullet\; P \squoted{*} F, 0)$ 
      \quad und\quad 
      $\pair(P \rightarrow \bullet\; F, 0)$. 
      \\[0.2cm]
      Wenden wir auf das Earley-Objekt $\pair(P \rightarrow \bullet\; F, 0)$
      die Vorhersage-Operation an, so erhalten wir schie{\ss}lich noch die folgenden Earley-Objekte in $Q_0$:
      \\[0.2cm]
      \hspace*{1.3cm}
      $\pair(F \rightarrow \bullet \squoted{1}, 0)$, \quad 
      $\pair(F \rightarrow \bullet \squoted{2}, 0)$, \quad und \quad
      $\pair(F \rightarrow \bullet \squoted{3}, 0)$. 
      \\[0.2cm]
      Insgesamt enth\"alt $Q_0$ nun die folgenden Earley-Objekte:
      \begin{enumerate}
      \item $\pair(\widehat{S} \rightarrow \bullet\; E, 0)$,
      \item $\pair(E \rightarrow \bullet\; E \squoted{+} P, 0)$
      \item $\pair(E \rightarrow \bullet\; P, 0)$,
      \item $\pair(P \rightarrow \bullet\; P \squoted{*} F, 0)$,
      \item $\pair(P \rightarrow \bullet\; F, 0)$,
      \item $\pair(F \rightarrow \bullet \squoted{1}, 0)$,
      \item $\pair(F \rightarrow \bullet \squoted{2}, 0)$,
      \item $\pair(F \rightarrow \bullet \squoted{3}, 0)$.
      \end{enumerate}

      Jetzt wenden wir die Lese-Operation auf $Q_0$ an.  Da das erste Zeichen des zu parsenden Strings eine
      ``1'' ist, hat die Menge  $Q_1$ danach die folgende Form:
      \\[0.2cm]
      \hspace*{1.3cm}
      $Q_1 = \bigl\{ \pair(F \rightarrow \squoted{1} \bullet, 0) \bigr\}$.
\item Nun setzen wir $i= 1$ und wenden zun\"achst auf $Q_1$ die Vervollst\"andigungs-Operation an.
      Aufgrund des Earley-Objekts $\pair(F \rightarrow \squoted{1} \bullet, 0) $ in $Q_1$
      suchen wir in $Q_0$ ein Earley-Objekt, bei dem die Markierung ``$\bullet$'' vor der Variablen
      $F$ steht.  Wir finden das Earley-Objekt
      $\pair(P \rightarrow \bullet\; F, 0)$.  Daher f\"ugen wir nun
      $Q_1$ das Earley-Objekt
      \\[0.2cm]
      \hspace*{1.3cm}
      $\pair(P \rightarrow F \;\bullet, 0)$ 
      \\[0.2cm]
      hinzu.  Hierauf k\"onnen wir wieder die
      Vervollst\"andigungs-Operation anwenden und finden (nach mehrmaliger Anwendung) f\"ur $Q_1$ insgesamt die folgenden Earley-Objekte:
      \begin{enumerate}
      \item $\pair(P \rightarrow F \;\bullet, 0)$, 
      \item $\pair(P \rightarrow  P\;\bullet \squoted{*} F, 0)$, 
      \item $\pair(E \rightarrow P \; \bullet, 0)$,
      \item $\pair(E \rightarrow E\;\bullet \squoted{+} P, 0)$,
      \item $\pair(\widehat{S} \rightarrow  E\;\bullet, 0)$.
      \end{enumerate}
      
      Als n\"achstes wenden wir auf diese Earley-Objekte die Vorhersage-Operation an.  Da das
      Markierungs-Zeichen ``$\bullet$'' aber in keinem der in $Q_i$ auftretenden Earley-Objekte vor einer 
      Variablen steht, ergeben sich hierbei keine neuen Earley-Objekte.

      Als letztes wenden wir die Lese-Operation auf $Q_1$ an.  Da in dem String
      ``\texttt{1+2*3}'' das Zeichen ``\texttt{+}'' an der Position 2 liegt ist und $Q_1$ das Earley-Objekt 
      \\[0.2cm]
      \hspace*{1.3cm}
      $\pair(E \rightarrow E\;\bullet \squoted{+} P, 0)$
      \\[0.2cm]
      enth\"alt, f\"ugen wir in $Q_2$  das Earley-Objekt
      \\[0.2cm]
      \hspace*{1.3cm}
      $\pair(E \rightarrow E \squoted{+}\bullet\; P, 0)$
      \\[0.2cm]
      ein.
\item Nun setzen wir $i= 2$ und wenden zun\"achst auf $Q_2$ die Vervollst\"andigungs-Operation
      an.  Zu diesem Zeitpunkt gilt
      \\[0.2cm]
      \hspace*{1.3cm}
      $Q_2 = \{ \pair(E \rightarrow E \squoted{+}\bullet\; P, 0) \}$.
      \\[0.2cm]
      Da in dem einzigen Earley-Objekt, das hier auftritt, das Markierungs-Zeichen ``$\bullet$''
      nicht am Ende der Grammatik-Regel steht, finden wir durch die Vervollst\"andigungs-Operation in
      diesem Schritt keine       neuen Earley-Objekte. 

      Als n\"achstes wenden wir auf $Q_2$ die Vorhersage-Operation an.  Da das Markierungs-Zeichen
      vor der Variablen $P$ steht, finden wir zun\"achst die beiden Earley-Objekte
      \\[0.2cm]
      \hspace*{1.3cm}
      $\pair(P \rightarrow \bullet\; F, 2)$ \quad und \quad
      $\pair(P \rightarrow \bullet\; P \squoted{*} F, 2)$.
      \\[0.2cm]
      Da in dem ersten Earley-Objekt das Markierungs-Zeichen vor der Variablen $F$ steht, kann 
      die Vorhersage-Operation ein weiteres 
      Mal angewendet werden und wir finden noch die folgenden Earley-Objekte:
      \begin{enumerate}
      \item $\pair(F \rightarrow \bullet \squoted{1}, 2)$, 
      \item $\pair(F \rightarrow \bullet \squoted{2}, 2)$,
      \item $\pair(F \rightarrow \bullet \squoted{3}, 2)$.
      \end{enumerate}

      Als letztes wenden wir die Lese-Operation auf $Q_2$ an.  Da das dritte Zeichen in dem zu lesenden
      String ``\texttt{1+2*3}'' die Ziffer ``2'' ist, hat $Q_3$ nun die Form
      \\[0.2cm]
      \hspace*{1.3cm}
      $Q_3 = \{ \pair(F \rightarrow \squoted{2}\bullet, 2) \}$.
\item Wir setzen $i = 3$ und wenden auf $Q_3$ die Vervollst\"andigungs-Operation an.
      Dadurch f\"ugen wir 
      \\[0.2cm]
      \hspace*{1.3cm}
      $\pair(P \rightarrow F \;\bullet, 2)$ 
      \\[0.2cm]
      in $Q_3$ ein.  Hier k\"onnen wir ein weiteres Mal die Vervollst\"andigungs-Operation anwenden. 
      Durch iterierte Anwendung der Vervollst\"andigungs-Operation erhalten wir zus\"atzlich die folgenden 
      Earley-Objekte:
      \begin{enumerate}
      \item $\pair(P \rightarrow P \bullet \squoted{*} F, 2)$,
      \item $\pair(E \rightarrow E \squoted{+} P\;\bullet, 0)$,
      \item $\pair(E \rightarrow E\;\bullet \squoted{+} P, 0)$
      \item $\pair(\widehat{S} \rightarrow E\;\bullet, 0)$.
      \end{enumerate}
      Als letztes wenden wir die Lese-Operation an.  Da der n\"achste zu lesende Buchstabe das Zeichen
      ``\texttt{*}'' ist, erhalten wir
      \\[0.2cm]
      \hspace*{1.3cm}
      $Q_4 = \{ \pair(P \rightarrow P \squoted{*} \bullet\, F, 2) \}$.
\item Wir setzen $i= 4$.  Die Vervollst\"andigungs-Operation liefert keine neuen Earley-Objekte.
      Die Vorhersage-Operation liefert folgende Earley-Objekte:
      \begin{enumerate}
      \item $\pair(F \rightarrow \bullet \squoted{1}, 4)$, 
      \item $\pair(F \rightarrow \bullet \squoted{2}, 4)$,
      \item $\pair(F \rightarrow \bullet \squoted{3}, 4)$.
      \end{enumerate}
      Da das n\"achste Zeichen die Ziffer ``3'' ist, liefert die Lese-Operation f\"ur $Q_5$:
      \\[0.2cm]
      \hspace*{1.3cm}
      $Q_5 = \pair(F \rightarrow \squoted{3}\bullet, 4) \}$.
\item Wir setzen $i=5$.  Die Vervollst\"andigungs-Operation liefert nacheinander die folgenden
      Earley-Objekte:
      \begin{enumerate}
      \item $\pair(P \rightarrow P \squoted{*} F\;\bullet, 2)$,
      \item $\pair(E \rightarrow E \squoted{+} P\;\bullet, 0)$,
      \item $\pair(P \rightarrow  P \;\bullet\squoted{*} F, 2)$,
      \item $\pair(E \rightarrow E \;\bullet\squoted{+} P, 0)$ 
      \item $\pair(\widehat{S} \rightarrow  E\;\bullet, 0)$.
      \end{enumerate}
\end{enumerate}
Da die Menge $Q_5$ das Earley-Objekt $\pair(\widehat{S} \rightarrow  E\;\bullet, 0)$ enth\"alt,
k\"onnen wir schlie{\ss}en, dass der String ``\texttt{1+2*3}'' tats\"achlich in der von der Grammatik erzeugten
Sprache liegt.
\vspace*{0.3cm}

\exercise
Zeigen Sie, dass der String ``\texttt{1*2+3}'' in der Sprache der Grammatik liegt, die in 
Abbildung \ref{fig:expr-small} gezeigt wird.  Benutzen Sie dazu den von Earley
angegebenen Algorithmus.

\section{Implementing Earley's Algorithm in \textsl{Python}}
The \textsl{Jupyter} notebook
\\[0.2cm]
\hspace*{-0.3cm}
\href{https://github.com/karlstroetmann/Formal-Languages/blob/master/ANTLR4-Python/Earley-Parser/Earley-Parser.ipynb}{https://github.com/karlstroetmann/Formal-Languages/blob/master/ANTLR4-Python/Earley-Parser/Earley-Parser.ipynb}
\\[0.2cm]
contains an implementation of Earley's algorithm.

\section{Check Your Understandig}
\begin{enumerate}[(a)]
\item Was ist ein Earley-Objekt und wie werden die Komponenten eines Earley-Objekts interpretiert?
\item Wie funktioniert die Lese-Operation beim Algorithmus von Earley?
\item Wie funktioniert die Vorhersage-Operation beim Algorithmus von Earley?
\item Wie funktioniert die Vervollst\"andigungs-Operation beim Algorithmus von Earley?
\item Skizzieren Sie den Algorithmus von Earley f\"ur den Fall, dass die Grammatik keine $\varepsilon$-Regeln enth\"alt.
\item Welche Komplexit\"at hat der Algorithmus von Earley im allgemeinen Fall?
\item Welche Komplexit\"at hat der Algorithmus von Earley in dem Fall, dass die Grammatik eindeutig ist?
\end{enumerate}


%%% Local Variables: 
%%% mode: latex
%%% TeX-master: "formal-languages"
%%% End: 
